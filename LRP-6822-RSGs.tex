%% Author: LRP
%% Date of last edit: 02-12-2014
%% Current version: LRP-6822-RSG.tex

%% The first piece of markup in an AASTeX v5.x document
%% is the \documentclass command. LaTeX will ignore
%% any data that comes before this command.

%% The command below calls the preprint style
%% which will produce a one-column, single-spaced document.
%% Examples of commands for other substyles follow. Use
%% whichever is most appropriate for your purposes.
%%
% \documentclass[12pt,preprint]{aastex}

%% manuscript produces a one-column, double-spaced document:

% \documentclass[manuscript]{aastex}
\documentclass[iop]{emulateapj}
\usepackage{natbib}
\citestyle{aa}
%% preprint2 produces a double-column, single-spaced document:

%% \documentclass[preprint2]{aastex}

%% Sometimes a paper's abstract is too long to fit on the
%% title page in preprint2 mode. When that is the case,
%% use the longabstract style option.

%% \documentclass[preprint2,longabstract]{aastex}

%% If you want to create your own macros, you can do so
%% using \newcommand. Your macros should appear before
%% the \begin{document} command.
%%
%% If you are submitting to a journal that translates manuscripts
%% into SGML, you need to follow certain guidelines when preparing
%% your macros. See the AASTeX v5.x Author Guide
%% for information.

\newcommand{\vdag}{(v)^\dagger}
\newcommand{\myemail}{lrp@roe.ac.uk}
\newcommand{\mdot}{\ensuremath{\dot{M}}}
\newcommand{\msun}{\ensuremath{M_{\odot}}}
\newcommand{\vsini}{\ensuremath{v_{\rm R} \sin i}}
\newcommand{\vrot}{\ensuremath{v_{\rm R}}}

\def\5{\footnotesize V\normalsize}
\def\4{\footnotesize IV\normalsize}
\def\3{\footnotesize III\normalsize}
\def\2{\footnotesize II\normalsize}
\def\1{\footnotesize I\normalsize}
\def\lam{$\lambda$}
\def\kms{$\mbox{km s}^{-1}$}
\def\p{$\phantom{:}$}
\def\a{$\phantom{^\ast}$}
\def\v{$\phantom{^{l}}$}
\def\pp{$\phantom{-}$}
\def\o{$\phantom{0}$}
\def\vr{$v_{\rm r}$}
%% You can insert a short comment on the title page using the command below.

% \slugcomment{Not to appear in Nonlearned J., 45.}

%% If you wish, you may supply running head information, although
%% this information may be modified by the editorial offices.
%% The left head contains a list of authors,
%% usually a maximum of three (otherwise use et al.).  The right
%% head is a modified title of up to roughly 44 characters.
%% Running heads will not print in the manuscript style.

\shorttitle{Red Supergiant Stars as Cosmic Abundance Probes}
\shortauthors{Patrick et al.}

%% This is the end of the preamble.  Indicate the beginning of the
%% paper itself with \begin{document}.

\begin{document}

%% LaTeX will automatically break titles if they run longer than
%% one line. However, you may use \\ to force a line break if
%% you desire.

\title{Red Supergiant Stars as Cosmic Abundance Probes: \\
    KMOS Observations in NGC\,6822}

%% Use \author, \affil, and the \and command to format
%% author and affiliation information.
%% Note that \email has replaced the old \authoremail command
%% from AASTeX v4.0. You can use \email to mark an email address
%% anywhere in the paper, not just in the front matter.
%% As in the title, use \\ to force line breaks.

\author{L.~R.~Patrick\altaffilmark{1},
C.~J.~Evans\altaffilmark{2,1},
B.~Davies\altaffilmark{3},
R-P.~Kudritzki\altaffilmark{4,5},
J.~Z.~Gazak\altaffilmark{4},
M.~Bergemann\altaffilmark{6},
B.~Plez\altaffilmark{7},
A.~M.~N.~Ferguson\altaffilmark{1}}
% \affil{Institute for Astronomy, University of Edinburgh, Royal Observatory Edinburgh, Blackford Hill, Edinburgh EH9 3HJ}

% \author{C. D. Biemesderfer\altaffilmark{4,5}}
% \affil{National Optical Astronomy Observatories, Tucson, AZ 85719}
% \email{aastex-help@aas.org}

% \and

% \author{R. J. Hanisch\altaffilmark{5}}
% \affil{Space Telescope Science Institute, Baltimore, MD 21218}

%% Notice that each of these authors has alternate affiliations, which
%% are identified by the \altaffilmark after each name.  Specify alternate
%% affiliation information with \altaffiltext, with one command per each
%% affiliation.

\altaffiltext{1}{Institute for Astronomy, University of Edinburgh, Royal Observatory Edinburgh, Blackford Hill, Edinburgh EH9 3HJ, UK}
\altaffiltext{2}{UK Astronomy Technology Centre, Royal Observatory Edinburgh, Blackford Hill, Edinburgh EH9 3HJ, UK}
\altaffiltext{3}{Astrophysics Research Institute, Liverpool John Moores University, Liverpool Science Park ic2, 146 Brownlow Hill, Liverpool L3 5RF, UK}
\altaffiltext{4}{Institute for Astronomy, University of Hawaii, 2680 Woodlawn Drive, Honolulu, HI, 96822, USA}
\altaffiltext{5}{University Observatory Munich, Scheinerstr. 1, D-81679 Munich, Germany}
\altaffiltext{6}{Institute of Astronomy, University of Cambridge, Madingley Road, Cambridge CB3 0HA, UK}
\altaffiltext{7}{Laboratoire Univers et Particules de Montpellier, Universit\'e Montpellier 2, CNRS, F-34095 Montpellier, France}
%% Mark off your abstract in the ``abstract'' environment. In the manuscript
%% style, abstract will output a Received/Accepted line after the
%% title and affiliation information. No date will appear since the author
%% does not have this information. The dates will be filled in by the
%% editorial office after submission.

\begin{abstract}
We present near-IR spectroscopy of red supergiant (RSG) stars in NGC\,6822, obtained with the new VLT-KMOS instrument.
From comparisons with model spectra in the $J$-band we determine the metallicity of 11 RSGs, finding a mean value of [Z] $= -0.52 \pm 0.21$ which agrees well with previous abundance studies of young stars and HII regions.
We also find an indication for a low-significance abundance-gradient within the central 1\,kpc.
We compare our results to those derived from older stellar populations and investigate the difference using chemical evolution models.
By comparing the physical properties determined for RSGs in NGC\,6822 with those derived using the same technique in the Galaxy and the Magellanic Clouds, we show that there appears to be no significant temperature variation of RSGs with respect to metallicity, in contrast to recent evolutionary models.
\end{abstract}

%% Keywords should appear after the \end{abstract} command. The uncommented
%% example has been keyed in ApJ style. See the instructions to authors
%% for the journal to which you are submitting your paper to determine
%% what keyword punctuation is appropriate.

\keywords{Galaxies: individual: NGC\,6822
-- stars: abundances
-- stars: supergiants}


\section{Introduction}

\label{sec:introduction}
A promising new method to directly probe chemical abundances in external galaxies is with $J$-band spectroscopy of red supergiant (RSG) stars.
With their peak flux at
$\sim$1\,$\mu$m and luminosities in excess of
10$^4$\,L$_\odot$, RSGs are extremely bright in the near-IR,
making them potentially useful tracers of the chemical abundances of star-forming galaxies out to large distances.
% (with $-$8\,$\le$\,M$_{J}$\,$\le$\,$-$11).
% Therefore, RSGs are useful tools with which to map the chemical evolution of their host galaxies, out to large distances.
To realise this goal,
\cite{2010MNRAS.407.1203D} outlined a technique to derive metallicities of RSGs at moderate spectral resolving power
($R\sim$3000).
This technique has recently been refined using observations of RSGs in the Magellanic Clouds
\citep{Davies-prep} and Perseus OB-1
\citep{2014ApJ...788...58G}.
Using absorption lines in the $J$-band from iron, silicon and titanium, one can estimate metallicity
([Z] = log (Z/Z$_{\odot}$)) as well as other stellar parameters
(effective temperature, surface gravity and microturbulence) by fitting synthetic spectra to the observations.
Owing to their intrinsic brightness,
RSGs are ideal candidates for studies of extragalactic environments in the near-IR.

To make full use of the potential of RSGs for this science, multi-object spectrographs operating in the near-IR on 8-m class telescopes are essential.
These instruments allow us to observe a large sample of RSGs in a given galaxy, at a wavelength where RSGs are brightest.
In this context, the K-band Multi-Object Spectrograph
\citep[KMOS;][]{2013Msngr.151...21S} at the Very Large Telescope (VLT), Chile, is a powerful facility.
KMOS will enable determination of stellar abundances for RSGs out to distances of $\sim$10\,Mpc.
Further ahead, a near-IR multi-object spectrograph on a 40-m class telescope, combined with the excellent image quality from adaptive optics,
will enable abundance estimates for individual stars in galaxies out to tens of Mpc,
a significant volume of the local universe containing entire galaxy clusters
\citep{2011A&A...527A..50E}.

Here we present KMOS observations of RSGs in the dwarf irregular galaxy NGC\,6822,
at a distance of $\sim$0.46\,Mpc
\citep[][and references therein]{2012AJ....144....4M}.
Chemical abundances have been determined for its old stellar population
\citep[e.g.][]{2001MNRAS.327..918T,2013ApJ...779..102K},
but knowledge of its recent chemical evolution and present-day abundances is
somewhat limited.
Observations of two A-type supergiants by
\cite{2001ApJ...547..765V} provided a first estimate of stellar abundances, finding
log (Fe/H) $+12=7.01\pm0.22$ and
log(O/H)   $+12=8.36\pm0.19$, based on line-formation calculations for these elements
assuming local thermodynamic equilibrium (LTE).
A detailed non-LTE study for one of these objects confirmed the results finding
$6.96\pm 0.09$ for iron and $8.30\pm0.02$ for oxygen
\citep{Przybilla02}.
Compared to solar values of 7.50 and 8.69, respectively
\citep{2009ARA&A..47..481A},
this indicates abundances that are approximately one third solar in NGC\,6822.
A study of oxygen abundances in HII regions
\citep{2006ApJ...642..813L} found a value of $8.11\pm0.1$, confirming the low metallicity.

NGC\, 6822 is a relatively isolated Local Group galaxy, which does not seem to be associated with either M31 or the Milky Way.
It appears to have a large extended stellar halo
\citep{2002AJ....123..832L,2014ApJ...783...49H}
as well as an extended HI disk containing tidal arms and a possible HI companion
\citep{2000ApJ...537L..95D}.
The HI disk is orientated perpendicular to the distribution of old halo stars and has an associated population of blue stars
\citep{2003MNRAS.341L..39D,2003ApJ...590L..17K}.
This led \cite{2006ApJ...636L..85D} to label the system as a
~\textquoteleft
polar ring galaxy\textquoteright.
A population of remote star clusters aligned with the elongated old stellar halo have been discovered
\citep{2011ApJ...738...58H,2013MNRAS.429.1039H}.
In summary, the extended structures of NGC\,6822 suggest some form of recent interaction.

In addition, there is evidence for a relatively constant star-formation history within the central 5\,kpc
\citep{2014ApJ...789..147W}
with multiple stellar populations
\citep{2006A&A...451...99B,2012A&A...540A.135S}.
This includes evidence for recent star formation in the form of a known population of massive stars, as well as a number of HII regions
\citep{2001ApJ...547..765V,2006AJ....131..343D,2009A&A...505.1027H,2012AJ....144....2L}.

In this paper we present near-IR KMOS spectroscopy of RSGs in NGC\,6822 to investigate their chemical abundances.
In Section~\ref{sec:observations} we describe the observations.
Section~\ref{sec:data_reduction} describes the data reduction and
Section~\ref{sec:results} details the derived stellar parameters and investigates the spatial distribution of the estimated metallicities in NGC\,6822.
In Section~\ref{sec:discussion} we discuss our results and
Section~\ref{sec:conclusions} concludes the paper.

% section introduction (end)

\section{Observations}
\label{sec:observations}

\subsection{Target Selection} % (fold)
\label{sub:target_selection}

Our targets were selected from optical photometry
\citep{2007AJ....134.2474M}, combined with near-IR ($JHK{_s}$) photometry
\cite[for details see][]{2012A&A...540A.135S} from the Wide-Field Camera (WFCAM) on the United Kingdom Infra-Red Telescope (UKIRT).
The two catalogues were cross-matched and only sources classified as stellar in the photometry for all filters were considered.

Our spectroscopic targets were selected principally based on their optical colours, as defined by
\cite{1998ApJ...501..153M} and
\cite{2012AJ....144....2L}.
Figure~\ref{fig:BVR} shows cross-matched stars, with the dividing line at $(B - V) = 1.25 \times (V - R)+0.45$.
All stars redder than this line, above a given magnitude threshold and with $V - R > 0.6$, are potential RSGs.

\textbf{Distinguishing between RSGs and the most luminous stars on the asymptotic giant branch
(AGB) is difficult owing to their similar temperatures and overlapping luminosities.
Near-IR photometry can help to delineate these populations,
with RSGs located in a relatively well-defined region in the near-IR ($J-K$) colour-magnitude diagram (CMD), as discussed by
\cite{2000ApJ...542..804N}.
The near-IR CMD from the WFCAM data is shown in
Figure~\ref{fig:JK} and was used to further inform our target selection.
Employing the updated CMD criteria from
\cite{2014A&A...562A..32C}  -- modified for the distance and reddening to NGC\,6822 --
all of our potential targets from the combined optical and near-IR criteria are (notionally) RSGs.}


The combined selection methods yielded 58 candidate RSGs, from which 18 stars were observed with KMOS, as shown in Figure
\ref{fig:N6822}.
The selection of the final targets was defined by the KMOS arm allocation software {\sc karma}
\citep{2008SPIE.7019E..0TW},
where the field centre was selected to maximise the number of allocated arms,
with priority given to the brightest targets.
Optical spectroscopy of eight of our observed stars, confirming them as RSGs, was presented by
\cite{2012AJ....144....2L}.


\begin{figure}
 \includegraphics[width=9.0cm]{figures/N6822_bvr}
 \caption{
          Two-colour diagram for stars with good detections in the optical and near-IR photometry in NGC\,6822.
          The black dashed line marks the selection criteria using optical colours, as defined by
          \protect\cite{2012AJ....144....2L}.
          Red circles mark all stars which satisfied our selection criteria.
           % and our $J$-$K$, $K$ cut.
          Large blue stars denote targets observed with KMOS.
          The solid black line marks the foreground reddening vector for E(B$-$V) = 0.22
          \protect\citep{1998ApJ...500..525S}.
         }
 \label{fig:BVR}
\end{figure}

\begin{figure}
 \includegraphics[width=9.0cm]{figures/N6822_jk}
 \caption{
          Near-IR colour-magnitude diagram (CMD) for stars classified as stellar sources in the optical and near-IR catalogues, plotted using the same symbols as Figure~\ref{fig:BVR}.
          This CMD is used to supplement the optical selection.
          The solid black line marks the foreground reddening vector for E(B$-$V) = 0.22
          \protect\citep{1998ApJ...500..525S}.
          % The green solid box indicates the region defined by
          % \protect\cite{2014A&A...562A..32C} as containing supergiants.
         }
 \label{fig:JK}
\end{figure}


\begin{figure}
 \includegraphics[width=9.0cm]{figures/N6822_RSGs_paper}
 \caption{Spatial extent of the KMOS targets over a Digital Sky Survey (DSS) image of NGC\,6822.
          Blue filled circles indicate the locations of the observed red supergiant stars.
          Red open circles indicate the positions of red supergiant candidates selected using out photometric criteria (see Section~\ref{sub:target_selection}).
          }
 \label{fig:N6822}
\end{figure}

% section target_selection (end)

\subsection{KMOS Observations} % (fold)
\label{sub:observations}

The observations were obtained as part of the KMOS Science Verification program on 30 June 2013 (PI: Evans, 60.A-9452(A)),
with a total exposure time of 2400\,s
(comprising 8\,$\times$\,300\,s detector integrations).
% Observations for this study are from the new KMOS instrument on the VLT, Chile.
KMOS has 24 deployable integral-field units (IFUs) each of which covers an area of
2\farcs8 $\times$ 2\farcs8 within a 7\farcm2 field-of-view.
The 24 IFUs are split into three groups of eight, with the light from each group relayed to different spectrographs.

Offset sky frames
(0\farcm5 to the east) were interleaved between the science observations in an object (O), sky (S) sequence of:
O,\,S,\,O,\,O.
The observations were performed with the $YJ$ grating
(giving coverage from 1.02 to 1.36$\mu$m);
estimates of the mean delivered resolving power for each spectrograph (obtained from the KMOS/esorex pipeline for two arc lines) are listed in Table~\ref{tb:res}.

In addition to the science observations, a standard set of KMOS calibration frames were obtained consisting of dark, flat and arc-lamp calibrations (with flats and arcs taken at six different rotator angles).
A telluric standard star was observed with the arms configured in the science positions, i.e. using the {\em KMOS\_spec\_cal\_stdstarscipatt} template in which the standard star is observed sequentially through all the IFUs.
The observed standard was HIP97618, with a spectral type of B6\,III
\citep{1988mcts.book.....H}.

A summary of the observed targets is given in
Table~\ref{tb:obs-params}.
A signal-to-noise (S/N) ratio of $\gtrsim$ 100 per resolution element is required for satisfactory results from this method
\citep[see][]{2014ApJ...788...58G}.
We estimated the S/N ratio of the spectra by comparing the source counts in the brightest spatial pixels
(within the 1.15-1.22$\mu$m region) of each source with the counts in equivalent spatial pixels in the corresponding sky exposures
(between the sky lines).
The S/N estimated is knowingly an underestimate of the true S/N achieved.

% \footnotetext{http://www.usm.uni-muenchen.de/people/wegner/kmos/en/karma.php}

\begin{table*}
\caption{Measured velocity resolution and resolving power across each detector.\label{tb:res}}
\scriptsize
\begin{center}
\begin{tabular}{crcccc}
\hline
\hline
Det. & IFUs & \multicolumn{2}{c}{Ne\,\lam1.17700\,$\mu$m}
            & \multicolumn{2}{c}{Ar\,\lam1.21430\,$\mu$m} \\
 & & FWHM [\kms] & $R$ & FWHM [\kms] & $R$ \\
  \hline
1 & 1-8 &  \a88.04\,$\pm$\,2.67 & 3\,408$\pm$\,103 &
           \o85.45\,$\pm$\,2.67 & 3\,511$\pm$\,110 \\
2 & 9-16 & \a82.83\,$\pm$\,2.48 & 3\,622$\pm$\,108 &
           \o80.30\,$\pm$\,3.05 & 3\,736$\pm$\,142 \\
3 & 17-24 & 103.23\,$\pm$\,2.73 & 2\,906$\pm$\,77\a &
            101.25\,$\pm$\,2.99 & 2\,963$\pm$\,87\a \\
\hline
\end{tabular}
\end{center}
\end{table*}


\begin{table*}
\caption{
        Summary of VLT-KMOS targets in NGC\,6822.
        \textbf{Spectral types and radial velocities included as per the referees comments.}\label{tb:obs-params}
        }
\scriptsize
\begin{center}
\begin{tabular}{lrcccccccccl}
 \hline
 \hline
ID & S/N & $\alpha$ (J2000) & $\delta$ (J2000) & $B$ & $V$ & $R$ & $J$ & $H$ & $K_{\rm s}$ & RV (\kms) & Notes \\
 \hline
NGC6822-RSG01 & 223 &   19:44:43.81  &  $-$14:46:10.7  &  20.83  &  18.59  &  17.23  &  14.16  &  13.37  &  13.09  & $-$69.5 $\pm$ 3.9 & Sample\\
NGC6822-RSG02 & 120 &   19:44:45.98  &  $-$14:51:02.4  &  20.91  &  18.96  &  17.89  &  15.53  &  14.72  &  14.52  & $-$59.2 $\pm$ 8.2 & Sample\\
NGC6822-RSG03 &  94 &   19:44:47.13  &  $-$14:46:27.1  &  21.30  &  19.41  &  18.41  &  16.13  &  15.35  &  15.12  & $-$50.4 $\pm$ 11.7 \\
NGC6822-RSG04 & 211 &   19:44:47.81  &  $-$14:50:52.5  &  20.74  &  18.51  &  17.22  &  14.37  &  13.58  &  13.30  & $-$65.0 $\pm$ 5.8 & LM12 (M1), Sample \\
NGC6822-RSG05 & 104 &   19:44:50.54  &  $-$14:48:01.6  &  20.83  &  18.95  &  17.97  &  15.75  &  14.98  &  14.79  & $-$66.0 $\pm$ 5.4 \\
NGC6822-RSG06 & 105 &   19:44:51.64  &  $-$14:48:58.0  &  21.33  &  19.45  &  18.32  &  15.81  &  14.95  &  14.72  & $-$199.2$\pm$ 49. \\
NGC6822-RSG07 & 145 &   19:44:53.46  &  $-$14:45:52.6  &  20.36  &  18.43  &  17.38  &  15.06  &  14.30  &  14.08  & $-$69.4 $\pm$ 4.0 & LM12 (M4.5), Sample \\
NGC6822-RSG08 & 103 &   19:44:53.46  &  $-$14:45:40.1  &  20.88  &  19.14  &  18.17  &  15.95  &  15.16  &  14.98  & $-$67.4 $\pm$ 4.8 & LM12 (K5), Sample \\
NGC6822-RSG09 & 201 &   19:44:54.46  &  $-$14:48:06.2  &  20.56  &  18.56  &  17.35  &  14.43  &  13.67  &  13.34  & $-$82.2 $\pm$ 1.2 & LM12 (M1), Sample\\
NGC6822-RSG10 & 302 &   19:44:54.54  &  $-$14:51:27.1  &  19.29  &  17.05  &  15.86  &  13.43  &  12.66  &  12.42  & $-$56.5 $\pm$ 9.3 & LM12 (M0), Sample \\
NGC6822-RSG11 & 327 &   19:44:55.70  &  $-$14:51:55.4  &  19.11  &  16.91  &  15.74  &  13.43  &  12.70  &  12.43  & $-$68.8 $\pm$ 4.2 & LM12 (M0), Sample \\
NGC6822-RSG12 & 100 &   19:44:55.93  &  $-$14:47:19.6  &  21.43  &  19.56  &  18.52  &  16.14  &  15.33  &  15.14  & $-$67.7 $\pm$ 4.7 & LM12 (K5) \\
NGC6822-RSG13 & 106 &   19:44:56.86  &  $-$14:48:58.5  &  21.05  &  19.06  &  18.04  &  15.81  &  15.05  &  14.85  & $-$74.6 $\pm$ 1.9 \\
NGC6822-RSG14 & 284 &   19:44:57.31  &  $-$14:49:20.2  &  19.69  &  17.41  &  16.20  &  13.52  &  12.76  &  12.52  & $-$49.3 $\pm$ 12.2 & LM12 (M1), Sample \\
NGC6822-RSG15 & 124 &   19:44:59.14  &  $-$14:47:23.9  &  21.30  &  19.17  &  18.05  &  15.58  &  14.74  &  14.50  & $-$45.7 $\pm$ 13.7 \\
NGC6822-RSG16 & 107 &   19:45:00.24  &  $-$14:47:58.9  &  21.27  &  19.20  &  18.10  &  15.60  &  14.80  &  14.57  & $-$65.4 $\pm$ 5.6 \\
NGC6822-RSG17 & 167 &   19:45:00.53  &  $-$14:48:26.5  &  20.84  &  18.75  &  17.51  &  14.70  &  13.86  &  13.61  & $-$62.2 $\pm$ 7.0 & Sample\\
NGC6822-RSG18 & 104 &   19:45:06.98  &  $-$14:50:31.1  &  21.06  &  19.12  &  18.06  &  15.74  &  14.94  &  14.78  & $-$74.5 $\pm$ 1.9 & Sample\\

\hline
\end{tabular}
\end{center}
\tablecomments{Optical data from
\protect\cite{2007AJ....134.2474M}, near-IR data from the UKIRT survey
\protect\cite[see][for details]{2012A&A...540A.135S}.
Targets observed by
\protect\cite{2012AJ....144....2L}
are indicated by \textquoteleft
LM12\textquoteright~ in the final column (with their spectral classifications in parentheses).
Targets used for abundance analysis are indicated by the comment
\textquoteleft Sample\textquoteright
.}
\end{table*}


% subsection KMOS observations (end)
% section observations (end)

\section{Data Reduction} % (fold)
\label{sec:data_reduction}

The observations were reduced using the recipes provided by the Software Package for Astronomical Reduction with KMOS
\citep[SPARK;][]{2013A&A...558A..56D}.
The standard KMOS/esorex routines were used to calibrate and reconstruct the science and standard-star data cubes as outlined by
\cite{2013A&A...558A..56D}.
Sky subtraction was performed using the standard KMOS recipes and telluric correction was performed using two different strategies.
Throughout the following analysis all spectra have been extracted from their respective data cubes using a consistent method (i.e. the optimal extractions within the pipeline).


\subsection{Three-arm vs 24-arm Telluric Correction} % (fold)
\label{sub:three_arm_vs_24_arm_telluric_correction}

The standard template for telluric observations with KMOS is to observe a standard star in one IFU in each of the three spectrographs.
However, there is an alternative template which allows users to observe a standard star in each of the 24 IFUs.
This strategy should provide an optimum telluric correction for the KMOS IFUs but reduces observing efficiency.

A comparison between the two methods in the $H$-band was given by
\cite{2013A&A...558A..56D},
who concluded that using the more efficient three-arm method was suitable for most science purposes.
However, an equivalent analysis in the $YJ$-band was not available.
To determine if the more rigorous telluric approach is required for our analysis,
we observed a telluric standard star (HIP97618) in each of the 24 IFUs.
This gave us the data to investigate both telluric correction methods and to directly compare the results.

We first compared the standard-star spectrum in each IFU with that used by the pipeline routines for the three-arm template in each of the spectrographs.
Figure~\ref{fig:IFU_compare} shows the differences between the standard-star spectra across the IFUs,
where the differences in the $YJ$-band are comparable to those in the $H$-band
\cite[cf. Fig.7 from][]{2013A&A...558A..56D}.
The qualitative agreement between the IFUs in our region of interest (1.15-1.22$\mu$m) is generally very good.


\begin{figure*}
 \begin{center}
 \includegraphics[width=12.0cm]{figures/N6822_t_compare}
 \caption{
    Comparison of $J$-band spectra of the same standard star in each IFU.
    The ratio of each spectrum compared to that from the IFU used in the three-arm telluric method is shown,
    with their respective mean and standard deviation ($\mu$ and $\sigma$).
    Red lines indicate $\mu$~=~1.0, $\sigma$~=~0.0 for each ratio.
    The blue shaded area signifies the region used in our analysis,
    within which, the discrepancies between the IFUs are generally small.
    This is reflected in the standard deviation values when only considering this region.
    (IFUs 13 and 16 are omitted as no data were taken with these IFUs.) \label{fig:IFU_compare}
          }
 \end{center}
\end{figure*}

To quantify the difference the two telluric methods would make to our analysis,
we performed the steps described in
Section~\ref{sub:ngc6822_telluric_correction} for both templates.
We then used the two sets of reduced science data
(reduced with both methods of the telluric correction) to compute stellar parameters for our targets.
The results of this comparison are detailed in Section
\ref{sub:telluric_comparison}.

% subsection three_arm_vs_24_arm_telluric_correction (end)

\subsection{Telluric Correction Implementation} % (fold)
\label{sub:ngc6822_telluric_correction}

To improve the accuracy of the telluric correction,
for both methods mentioned above,
we implemented additional recipes beyond those of the KMOS/esorex pipeline.
These recipes were employed to account for two effects which could potentially degrade the quality of the telluric correction.
The first corrects for any potential shift in wavelength between each science spectrum and its associated telluric standard.
The most effective way to implement this is to cross-correlate each pair of science and telluric standard spectra.
Any shift between the two is then applied to the telluric standard using a cubic-spline interpolation routine.

The second correction applied is a simple spectral scaling algorithm.
This routine corrects for differences in line intensity of the most prominent features common to both the telluric and science spectra.
To find the optimal scaling parameter the following formula is used,

\begin{equation} \label{eq:shiftandres}
T_{2} = (T_{1} + c) / (T_{1} - c),
\end{equation}

\noindent where $T_{2}$ is the corrected telluric standard spectrum,
$T_{1}$ is the initial telluric standard spectrum and $c$ is the scaling parameter.

To determine the required scaling,
telluric spectra are computed for $-0.5 < c < 0.5$, in increments of 0.02
(where a perfect value, i.e. no difference in line strength would be $c = 0$).
Each telluric spectrum is used to correct the science data and the standard deviation of the counts across the spectral region is computed for each corrected spectrum.
The minimum value of the standard-deviation matrix defines the optimum scaling.
For this algorithm, only the region of interest for our analysis is considered
(i.e. 1.15-1.22$\mu$m).

The final set of telluric-standard spectra,
from the KMOS/esorex reductions were modified using these additional routines and were then used to correct the science observations for the effects of the Earth's atmosphere.

% subsection ngc6822_telluric_correction (end)


\begin{figure*}
 %\vspace{302pt}
 \begin{center}
\includegraphics[width=16cm]{figures/N6822_mod_fit.pdf}
\caption{KMOS spectra of the NGC\,6822 RSGs and their associated best-fit model spectra
(black and red lines, respectively).
The lines used for the analysis from left-to-right by species are:
Fe\,I$\lambda\lambda$1.188285,
1.197305,
Si\,I$\lambda\lambda$1.198419,
1.199157,
1.203151,
1.210353,
Ti\,I$\lambda\lambda$1.189289,
1.194954.
The two strong Mg\,I lines are also labelled, but are not used in the fits
(see Section~\ref{sec:results})
Some of the strongest lines are marked.
         }
\label{fig:model_fits}
\end{center}
\end{figure*}


\subsection{{\sc molecfit}} % (fold)
\label{sub:molecfit}

As an alternative to observing telluric standard stars, a new telluric correction package, {\sc molecfit}, allows one to calculate a telluric spectrum based on atmospheric modelling.
Briefly, the software uses a reference atmospheric profile to estimate the true profile for the time and location of the science observation.
This model is then used to create a telluric spectrum which can be used to correct the observations.

This software has been shown to work well, on a variety of VLT instruments
(Smette et al. submitted) and has been rigorously tested using X-shooter spectra
(Kausch et al. submitted).
However, the package has yet to be tested thoroughly on lower-resolution observations such as those from KMOS.
Our first tests appear encouraging but, pending further characterisation of the KMOS data cubes
(e.g., small variations in spectral resolving power leading to sky residuals,
see Section~\ref{sub:sky_subtraction}),
we will investigate the potential of the {\sc molecfit} package in future papers of this series.

% subsection molecfit(end)

\subsection{Sky Subtraction} % (fold)
\label{sub:sky_subtraction}

Initial inspection of the extracted stellar spectra revealed minor residuals from the sky subtraction process.
Reducing these cases with the \textquoteleft sky\_tweak\textquoteright
~option within the KMOS/esorex reduction pipeline was ineffective to improve the subtraction of these features.
Any residual sky features could potentially influence our results by perturbing the continuum placement within the model fits, which is an important aspect of the fitting process
\citep[see][for more discussion]{2014ApJ...788...58G,Davies-prep}.
Thus, pending a more rigorous treatment of the data
(e.g. to take into account the changing spectral resolution across the array),
we exclude objects showing sky residuals from our analysis.
Of the 18 observed targets, 11 were used to derive stellar parameters
(as indicated in Table~\ref{tb:obs-params}).

% subsection sky_subtraction (end)
\subsection{Stellar Radial Velocities} % (fold)
\label{sub:RVs}

\textbf{
  Radial velocities for each target are listed in Table~\ref{tb:obs-params}.
  To increase confidence in the wavelength solution provided by the data reduction pipeline,
  this solution is cross-correlated onto the wavelength solution of a spectum of the Earth's telluric features.
  In general, this is a small correction.
  The science spectra are then telluric corrected in the manor described above.}

\textbf{
  Radial velocities are derived by a cross-correlation between the rest frame telluric-corrected science spectra
  (using only the brightest spectral pixel) and a synthetic RSG spectrum.
  The errors are calculated by taking the dispersion of each line used to derive the radial velocities scaled by the number of lines used following
  \cite{2014arXiv1410.5825L}.
  A robust radial velocity could not be derived for one of our candidates
  (NGC6822RSG21) owing to strong residual sky subtraction features.
  Therefore, this target it not taken into account for any further analysis.}

\textbf{
  Derived radial velocities are shown as a function of distance to the centre of NGC\,6822 in Figure~\ref{fig:RvsRV}.
  The average radial velocity for our targets is $-67.6\pm12.1$\,\kms.
  This is in good agreement with the systemic radial velocity of the H\,I disk
  \citep[$-57\pm2$\,\kms;][]{2004AJ....128...16K}.
  Our radial velocities are also in agreement with two A-type supergiants in NGC\,6822
  \citep{2001ApJ...547..765V}.
  This result confirms our candiadates are all within NGC\,6822 as well as their classification as supergiants.}

\begin{figure}
\includegraphics[width=9.0cm]{figures/N6822_RvsRV}
\caption{
Radial velocities of targets shown against their distance from the galaxy centre.
The average radial velocity for our sample is $-67.6\pm12.1$\,\kms.
The green dashed line indicates H\,I systemic velocity
\protect\citep[$-57\pm2$\,\kms;][]{2004AJ....128...16K}.
The radial velocities of two A-type supergiants from
\protect\cite{2001ApJ...547..765V} are shown in blue.
        }
\label{fig:RvsRV}
\end{figure}

% subsection RVs (end)
% section data_reduction_and_analysis (end)


\section{Results} % (fold)
\label{sec:results}

Stellar parameters
(metallicity, effective temperature, surface gravity and microturbulence)
have been derived using the $J$-band analysis technique described by
\cite{2010MNRAS.407.1203D} and demonstrated by
\cite{Davies-prep} and
\cite{2014ApJ...788...58G}.
To estimate physical parameters this technique uses a grid of synthetic spectra to fit observational data,
in which the models are degraded to the resolution of the observed spectra
(Table~\ref{tb:res}).
Model atmospheres were generated using the {\sc marcs} code
\citep{2008A&A...486..951G} where the range of parameters are defined in
Table~\ref{tb:mod_range}.
The precision of the models is increased by including departures from LTE in some of the strongest Fe, Ti and Si atomic lines
\citep{2012ApJ...751..156B,2013ApJ...764..115B}.
The two strong magnesium lines in our diagnostic spectral region are excluded from the analysis at present as these lines are known to be affected strongly by non-LTE effects
(see Figure~\ref{fig:model_fits}, where the two MgI lines are systematically under- and over-estimated, respectively).
The non-LTE effects on the formation of the Mg I lines will be explored by
Bergemann et al. (in prep).

\begin{table}
\caption{
Model grid used for analysis.\label{tb:mod_range}
         }
\scriptsize
\begin{center}
\begin{tabular}{lccc}
 \hline
 \hline
  Model Parameter & Min. & Max. & Step size \\
 \hline
T$_{eff}$ (K)        & 3400 & 4000 & 100 \\
                     & 4000 & 4400 & 200 \\
$[$Z$]$ (dex)   & $-$1.50 & 1.00  & 0.25\\
log $g$ (cgs)  & $-$1.0\o & 1.0\o & 0.5\o \\
 $\xi$ (\kms)  & \pp1.0\o & 6.0\o & 1.0\o\\
 \hline
\end{tabular}
\end{center}
\end{table}

% subsection telluric_comparison (end)

\subsection{Telluric Comparison} % (fold)
\label{sub:telluric_comparison}

We used these Science Verification data to determine which of the two telluric standard methods is most appropriate for our analysis.
Table~\ref{tb:stellar-params} details the stellar parameters derived for each target using both telluric methods and these parameters are compared in
Figure~\ref{fig:3vs24AT}.
The mean difference in metallicity from the two methods is
$\Delta [Z] = 0.04 \pm 0.07$.
Therefore, for our analysis, there is no significant difference between the two telluric approaches.


\begin{figure*}
 \begin{center}$
  \centering
  \begin{array}{cc}
  \includegraphics[width=9.0cm]{figures/N6822_24vs3AT_Z} &
  \includegraphics[width=9.0cm]{figures/N6822_24vs3AT_Teff} \\
  \includegraphics[width=9.0cm]{figures/N6822_24vs3AT_logg} &
  \includegraphics[width=9.0cm]{figures/N6822_24vs3AT_Xi} \\
  \end{array}$
 \end{center}
 \caption{
            Comparison of the model parameters using the two different telluric methods.
            In each panel, the x-axis represents stellar parameters derived using the 3-arm telluric
            method and the y-axis represents those derived using the 24-arm telluric method.
            Top left: metallicity ([Z]), mean difference
            $<\Delta[Z]>~= 0.04 \pm 0.07$.
            Top right: effective temperature (T$_{\rm eff}$), mean difference
            $<\Delta $T$_{\rm eff}>~= -14 \pm 42$.
            Bottom left: surface gravity (log $g$), mean difference
            $<\Delta$ log\,$g>~= -0.06 \pm 0.12$.
            Bottom right: Microturbulence ($\xi$), mean difference
            $<\Delta \xi>~= -0.1 \pm 0.1$.
            Green dashed lines indicates linear best fits to the data.
            In all cases, the distributions are statistically consistent with a one-to-one ratio (black lines).
          }
 \label{fig:3vs24AT}
\end{figure*}


\begin{table*}
\begin{center}
\caption{
Fit parameters for reductions using the two different telluric methods.
\label{tb:stellar-params}
         }
\scriptsize
\begin{tabular}{lc cccc c cccc}
 \hline
 \hline
  Target  & IFU &  \multicolumn{4}{c}{24 Arm Telluric} & \multicolumn{4}{c}{3 Arm Telluric}\\
  \cline{3-6}  \cline{8-11}
 &  & T$_{eff}$ (K) & log g & $\xi$ (\kms) & [Z] & & T$_{eff}$ (K) & log g & $\xi$ (\kms) & [Z]\\
  \hline
NGC6822-RSG01 & 6 & 3790 $\pm$ 80\o & $-$0.0 $\pm$ 0.3 & 3.5 $\pm$ 0.4 & $-$0.55 $\pm$ 0.18 & & 3860 $\pm$ 90\o & $-$0.1 $\pm$ 0.5 &  3.5 $\pm$ 0.4 & $-$0.61 $\pm$ 0.21 \\
NGC6822-RSG02 & 11& 3850 $\pm$ 100  & \pp0.4 $\pm$ 0.5 & 3.5 $\pm$ 0.4 & $-$0.78 $\pm$ 0.22 & & 3810 $\pm$ 110  & \pp0.4 $\pm$ 0.5 &  3.3 $\pm$ 0.5 & $-$0.65 $\pm$ 0.24 \\
NGC6822-RSG04 & 12& 3880 $\pm$ 70\o & \pp0.0 $\pm$ 0.3 & 4.0 $\pm$ 0.4 & $-$0.32 $\pm$ 0.16 & & 3880 $\pm$ 70\o & \pp0.0 $\pm$ 0.3 &  4.0 $\pm$ 0.4 & $-$0.32 $\pm$ 0.16 \\
NGC6822-RSG07 & 2 & 3970 $\pm$ 60\o & \pp0.4 $\pm$ 0.5 & 3.9 $\pm$ 0.4 & $-$0.58 $\pm$ 0.19 & & 3990 $\pm$ 80\o & \pp0.1 $\pm$ 0.5 &  3.8 $\pm$ 0.5 & $-$0.56 $\pm$ 0.14 \\
NGC6822-RSG08 & 3 & 3910 $\pm$ 100  & \pp0.6 $\pm$ 0.5 & 3.0 $\pm$ 0.4 & $-$0.58 $\pm$ 0.24 & & 3910 $\pm$ 100  & \pp0.6 $\pm$ 0.5 &  3.0 $\pm$ 0.4 & $-$0.58 $\pm$ 0.24 \\
NGC6822-RSG09 & 4 & 3980 $\pm$ 60\o & \pp0.1 $\pm$ 0.4 & 3.7 $\pm$ 0.4 & $-$0.38 $\pm$ 0.16 & & 3990 $\pm$ 80\o & $-$0.1 $\pm$ 0.5 &  3.6 $\pm$ 0.4 & $-$0.44 $\pm$ 0.17 \\
NGC6822-RSG10 & 14& 3900 $\pm$ 80\o & $-$0.3 $\pm$ 0.5 & 3.7 $\pm$ 0.4 & $-$0.67 $\pm$ 0.16 & & 3850 $\pm$ 80\o & $-$0.3 $\pm$ 0.5 &  3.5 $\pm$ 0.4 & $-$0.59 $\pm$ 0.19 \\
NGC6822-RSG11 & 15& 3870 $\pm$ 80\o & $-$0.4 $\pm$ 0.5 & 4.2 $\pm$ 0.5 & $-$0.53 $\pm$ 0.19 & & 3850 $\pm$ 60\o & $-$0.3 $\pm$ 0.4 &  4.3 $\pm$ 0.5 & $-$0.49 $\pm$ 0.17 \\
NGC6822-RSG14 & 17& 3910 $\pm$ 110  & $-$0.5 $\pm$ 0.5 & 3.6 $\pm$ 0.5 & $-$0.20 $\pm$ 0.21 & & 3880 $\pm$ 110  & $-$0.5 $\pm$ 0.5 &  3.6 $\pm$ 0.5 & $-$0.15 $\pm$ 0.24 \\
NGC6822-RSG17 & 21& 3890 $\pm$ 120  & \pp0.1 $\pm$ 0.5 & 3.0 $\pm$ 0.4 & $-$0.43 $\pm$ 0.28 & & 3890 $\pm$ 120  & \pp0.1 $\pm$ 0.5 &  3.0 $\pm$ 0.4 & $-$0.43 $\pm$ 0.28 \\
NGC6822-RSG18 & 18& 3810 $\pm$ 130  & \pp0.4 $\pm$ 0.5 & 2.2 $\pm$ 0.4 & $-$0.68 $\pm$ 0.31 & & 3740 $\pm$ 130  & \pp0.4 $\pm$ 0.5 &  2.1 $\pm$ 0.5 & $-$0.54 $\pm$ 0.41 \\

% Stars not included in the sample:

% RSG9  & 5 & 4040 $\pm$ 130  & \pp0.8 $\pm$ 0.3 & 3.8 $\pm$ 0.6 & \pp0.14 $\pm$ 0.24 & 2500 & & 3990 $\pm$ 130  & \pp0.8 $\pm$ 0.3 &  3.8 $\pm$ 0.6 & \pp0.09 $\pm$ 0.24 & 2500 \\
% RSG16 & 7 & 4200 $\pm$ 50\o & \pp0.5 $\pm$ 0.4 & 2.3 $\pm$ 0.3 & $-$0.51 $\pm$ 0.16 & 4400 & & 4230 $\pm$ 100  & \pp0.5 $\pm$ 0.2 &  2.4 $\pm$ 0.3 & $-$0.54 $\pm$ 0.15 & 4400 \\
% RSG21 & 10& 3780 $\pm$ 110  & \pp0.4 $\pm$ 0.5 & 2.3 $\pm$ 0.4 & $-$0.55 $\pm$ 0.32 & 3800 & & 3790 $\pm$ 90\o & \pp0.5 $\pm$ 0.5 &  2.2 $\pm$ 0.4 & $-$0.46 $\pm$ 0.33 & 3800 \\
% RSG36 & 1 & 3840 $\pm$ 120  & \pp0.6 $\pm$ 0.5 & 3.0 $\pm$ 0.4 & $-$0.52 $\pm$ 0.26 & 3300 & & 3930 $\pm$ 100  & \pp0.6 $\pm$ 0.4 &  2.9 $\pm$ 0.4 & $-$0.39 $\pm$ 0.25 & 3300 \\
% RSG39 & 19& 3870 $\pm$ 130  & \pp0.5 $\pm$ 0.5 & 2.4 $\pm$ 0.5 & $-$0.73 $\pm$ 0.26 & 3000 & & 3800 $\pm$ 120  & \pp0.4 $\pm$ 0.5 &  2.1 $\pm$ 0.5 & $-$0.58 $\pm$ 0.33 & 3000 \\
% RSG45 & 24& 4220 $\pm$ 120  & \pp0.6 $\pm$ 0.5 & 2.9 $\pm$ 0.4 & $-$0.24 $\pm$ 0.20 & 3200 & & 4290 $\pm$ 120  & \pp0.6 $\pm$ 0.5 &  3.0 $\pm$ 0.5 & $-$0.27 $\pm$ 0.22 & 3200 \\
% RSG47 & 22& 3980 $\pm$ 90\o & \pp0.4 $\pm$ 0.4 & 3.2 $\pm$ 0.4 & $-$0.57 $\pm$ 0.23 & 2700 & & 3950 $\pm$ 100  & \pp0.4 $\pm$ 0.4 &  3.3 $\pm$ 0.4 & $-$0.64 $\pm$ 0.24 & 2700 \\

  \hline
  \end{tabular}
  \end{center}
\end{table*}

\subsection{Stellar Parameters and Metallicity} % (fold)
\label{sub:stellar_parameters_and_metallicity}

Table~\ref{tb:stellar-params} summarises the derived stellar parameters.
For the remainder of this paper, when discussing stellar parameters,
we adopt those derived using the 24-arm telluric method
(i.e. the results in the left-hand part of Table~\ref{tb:stellar-params}.)
The average metallicity for our sample of 11 RSGs in NGC\,6822 is
$\bar{Z} = -0.52\pm 0.21$.
This result is in good agreement with the average metallicity derived in
NGC\,6822 from blue supergiant stars
\citep[BSGs;][]{1999A&A...352L..40M,2001ApJ...547..765V}.

A direct comparison with metallicities from BSGs is legitimate as the results derived here yield a global metallicity ([Z]) which
closely resembles the Fe/H ratio derived in
\cite{2001ApJ...547..765V}.
While our [Z] measurements are also affected by Si and Ti,
we assume [Z]~=~[Fe/H] for the purposes of our discussion.
Likewise, we can compare oxygen abundances (relative to solar) obtained from HII regions as a proxy for [Z] by
introducing the solar oxygen abundance
{12+log(O/H)}$_{\odot}$=8.69
\citep{2009ARA&A..47..481A} through the relation
[Z]=12+log(O/H)$-$8.69.

The RSG and BSG stages are different evolutionary phases within the life cycle of a massive star,
while HII regions are the birth clouds which give rise to the youngest stellar population.
As the lifetimes of RSGs and BSGs are $<50$Myr,
their metallicity estimates are also expected to be representative of their birth clouds.

To investigate the spatial distribution of chemical abundances in NGC\,6822,
in Figure~\ref{fig:ZvsR}
we show the metallicities of our RSGs as a function of radial distance from the centre of the galaxy,
as well as the results from
\cite{2001ApJ...547..765V} and the indicative estimates from
\cite{1999A&A...352L..40M}.

A least-squares fit to the KMOS results reveals a low-significance abundance gradient within the central 1\,kpc of NGC\,6822 of $-0.5\pm0.4$ dex\,kpc$^{-1}$.
The extrapolated central metallicity from the fit (i.e. at R$=$0) of [Z]$=-0.30\pm0.15$ derived remains consistent with the average metallicity assuming no gradient.

\begin{figure}
\includegraphics[width=9.0cm]{figures/N6822_ZvsR_BSG}
\caption{
Derived metallicities for 11 RSGs in NGC\,6822 shown against their distance from the galaxy centre;
the average metallicity is
$\bar{Z} = -0.52\pm 0.21$.
Blue and red points show the results from blue supergiants from
\protect\cite{2001ApJ...547..765V} and
\protect\cite{2001ApJ...547..765V} respectively.
A least-squares fit to the KMOS results reveals a low-significance abundance gradient
(see text for details).
For comparison,
R$_{25} = 460\arcsec$
\citep[$=1.02$\,kpc;][]{2012AJ....144....4M}.
        }
% Blue points show the results from two A-type supergiant stars from
% \protect\cite{2001ApJ...547..765V}.
% A least-squares fit to the KMOS results reveals a low-significance abundance gradient
% (see text for details) and including these results from
% \protect\cite{2001ApJ...547..765V} gives consistent results.
% Red points show estimates from three BSG from
% \protect\cite{1999A&A...352L..40M}.
% Results from
% \protect\cite{1999A&A...352L..40M} are not included into the fit as these measurements are qualitative estimates of metallicity.
% For comparison,
% R$_{25} = 460\arcsec$
% \citep[$=1.02$\,kpc;][]{2012AJ....144....4M}.
\label{fig:ZvsR}
\end{figure}

Figure~\ref{fig:6822HRD} shows the location of our sample in the Hertzsprung-Russell (H-R) diagram.
Bolometric corrections were computed using the calibration in
\cite{2013ApJ...767....3D} to calculate luminosities.
This figure shows that the temperatures derived using the $J$-band method are systematically cooler than the end of the evolutionary models (which terminate at the end of the carbon-burning phase for massive stars) for
$Z=0.002$ from
\cite{2013A&A...558A.103G}.
This is discussed in Section~\ref{sub:temperatures_of_rsgs}.


\begin{figure}
\includegraphics[width=9.0cm]{figures/N6822_HRD}
\caption{
H-R diagram for the 11 RSGs in NGC\,6822.
Evolutionary tracks including rotation
($v/v_{c} = 0.4$) for SMC-like metallicity ($Z=0.002$)
are shown in grey, along with their zero-age mass
\protect\citep{2013A&A...558A.103G}.
Bolometric corrections are computed using the calibration in
\protect\cite{2013ApJ...767....3D}.
We note that compared to the evolutionary tracks,
the observed temperatures of NGC\,6822 RSGs are systematically cooler.
This is discussed in Section~\ref{sub:temperatures_of_rsgs}.
}
\label{fig:6822HRD}
\end{figure}

% subsection stellar_parameters (end)
% section results (end)


\section{Discussion} % (fold)
\label{sec:discussion}

\subsection{Metallicity Measurements} % (fold)
\label{sub:metallicity_measurements}

We find an average metallicity for our sample of $\bar{Z}=-0.52\pm 0.21$
which agrees well with the results derived from BSGs
\citep{1999A&A...352L..40M,2001ApJ...547..765V,Przybilla02} and HII regions
\citep{2006ApJ...642..813L}.

We also find evidence for a low-significance metallicity gradient within the central 1\,kpc of NGC\,6822
($-0.5\pm0.4$ dex\,kpc$^{-1}$; see Figure~\ref{fig:ZvsR}).
The gradient derived is consistent with the trend reported in
\cite{2001ApJ...547..765V}
from their results for the two BSGs compare with HII regions from
\cite{1980MNRAS.193..219P} and two planetary nebulae from
\cite{1995ApJ...445..642R} at larger galactocentric distances.
Our result is also consistent with the gradient derived from a sample of 49 local star-forming galaxies
(Ho et al. submitted).
Including the results for BSGs from
\cite{2001ApJ...547..765V}
in our analysis,
gives a consistent gradient
($-0.48\pm 0.33$ dex\,kpc$^{-1}$)
with a smaller
$\chi^{2}_{red}=1.06$.
Results from
\cite{1999A&A...352L..40M} are not included into the fit as these measurements are qualitative estimates of metallicity.

In contrast,
\cite{2006ApJ...642..813L} used the oxygen abundances from 19 HII
regions and found no clear evidence for a metallicity gradient.
Using a subset of the highest quality HII
region data available these authors found a gradient of
$-0.16\pm0.05$dex\,kpc$^{-1}$.
Including these results into our analysis degrades the fit and changes the derived gradient significantly
($-0.18\pm0.05$ dex\,kpc$^{-1}$; $\chi^{2}_{red}=1.78$).
At this point it is not clear whether the indication of a gradient obtained from the RSGs and BSGs is an artefact of the small sample size,
or indicates a difference with respect to the HII region study.


There have been a number of studies of the metallicity of the older stellar population in NGC\,6822.
From spectroscopy of red giant branch (RGB) stars,
\cite{2001MNRAS.327..918T} found a mean metallicity of [Fe/H] $=-0.9$
with a reasonably large spread (see their Figure 19).
More recently,
\cite{2012A&A...540A.135S} derived metallicities using a population of AGB stars within the central 4\,kpc of NGC\,6822.
They found an average metallicity of [Fe/H] $=-1.29\pm0.07$ dex.
Likewise,
\cite{2013ApJ...779..102K}
used spectra of red giant stars within the central 2\,kpc and found an average metallicity of
[Fe/H] $=-1.05\pm0.49$.
We note that none of these studies found any compelling evidence for spatial variations in the stellar metallicities,
which is not surprising given that radial migration is thought to smooth out any abundance gradients over time.
The stellar populations used for these studies are known to be significantly older than our sample,
therefore, owing to the chemical evolution in the time since the birth of older populations,
we expect the measured metallicities to be significantly lower.

The low metallicity of the young stellar population and the interstellar medium (ISM) in NGC 6822 can be easily understood as a consequence of the fact that it is a relatively gas-rich galaxy with a mass
M$_{HI}$ = 1.38$\times$10$^{8}$ M$_{\odot}$
\citep{2004AJ....128...16K} and a total stellar mass of
M$_{*}$ = 1$\times$10$^{8}$ M$_{\odot}$
\citep{2014ApJ...789..147W}.
% subsection abundance_measurements (end)

\subsection{A Closed-box Chemical Evolution Model for NGC\,6822} % (fold)
\label{sub:a_closed_box_chemical_evolution_model_for_ngc,6822}

The simple closed-box chemical-evolution model relates the metallicity mass fraction $Z(t)$ at any time to the ratio of stellar to gas mass $M_{*}\over M_{g}$ through

\begin{equation}\label{closed-box}
Z(t) = {y \over 1-R } \ln \left[ 1 + {M_{*}(t)\over M_{g}(t)}  \right],
\end{equation}

where $y$ is the fraction of metals per stellar mass produced through stellar nucleosynthesis
(the so-called yield) and $R$ is the fraction of stellar mass returned to the ISM through stellar mass-loss.

According to
Kudritzki et al. (in prep), the ratio $y/(1-R)$ can be empirically determined from the fact that the metallicity of the young stellar population in the solar neighbourhood is solar, with a mass fraction Z$_{\odot}$ =0.014
\citep{2012A&A...539A.143N}.
With a stellar-to-gas mass column density of 4.48 in the solar neighbourhood
\citep{2003ApJ...587..278W,2013ApJ...779..115B}
one then obtains $y/(1-R)$ = 0.0082 = 0.59Z$_{\odot}$ with an uncertainty of 15 percent dominated by the 0.05 dex uncertainty of the metallicity determination of the young population.

Accounting for the presence of helium and metals in the neutral interstellar gas we can turn the observed HI mass in NGC 6822 into a gas mass via
M$_{g}$ = 1.36 M$_{HI}$ and use the simple closed-box model to predict a metallicity of
[Z] = $-0.6\pm0.05$,
in good agreement with our value obtained from RSG spectroscopy.

As discussed above, the older stellar population of AGB stars has a metallicity roughly 0.7 dex lower than the RSGs. In the framework of the simple closed-box model this would correspond to a period in time where the ratio of stellar to gas mass was 0.07
(much lower than the present value of 0.53) and the stellar mass was only
$0.19\times10^{8}$M$_{\odot}$.
The present star-formation rate of NGC 6822 is 0.027 M$_{\odot}$yr$^{-1}$
\citep{1996A&A...308..723I,2006ApJ...652.1170C,2010A&A...512A..68G}.
At such a high level of star formation it would have taken three Gyr to produce the presently observed stellar mass and to arrive from the average metallicity of the AGB stars to that of the young stellar population
(of course, again relying on the simple closed-box model).
With a lower star-formation rate it would have taken correspondingly longer.

Given the irregularities present in the morphology of NGC\,6822, this galaxy may not be a good example of a closed-box system, however it is remarkable that the closed-box model reproduces the observed metallicity so closely.

% subsection a_closed_box_chemical_evolution_model_for_ngc,6822 (end)

\subsection{Temperatures of RSGs} % (fold)
\label{sub:temperatures_of_rsgs}

Effective temperatures have been derived for 11 RSGs from our observed sample in NGC\,6822.
To date, this represents the fourth data set used to derive stellar parameters in this way and the first with KMOS.
The previous three data sets which have been analysed are those of 11 RSGs in PerOB1,
a Galactic star cluster
\citep{2014ApJ...788...58G}, nine RSGs in the LMC and 10 RSGs in the SMC
\citep[both from][]{Davies-prep}.
These results span a range of $\sim$0.7 dex in metallicity ranging from Z=Z$_{\odot}$ in PerOB1 to Z=0.3Z$_{\odot}$ in the SMC.

We compare the effective temperatures derived in this study to those of the previous results in different environments.
Their distribution is shown as a function of metallicity in Figure~\ref{fig:TvsZ}.
Additionally, Figure~\ref{fig:HRD} shows the H-R diagram for the four sets of results.
Bolometric corrections to calculate the luminosities for each sample were computed using the calibration in
\cite{2013ApJ...767....3D}.


\begin{figure}
\includegraphics[width=9.0cm]{figures/N6822_TeffvsZ_all}
\caption{
Effective temperatures as a function of metallicity for four different data sets using the $J$-band analysis technique.
There appears to be no significant variation in the temperatures of RSGs over a range of 0.7 dex.
These results are compiled from the LMC, SMC
\protect\citep[blue and red points respectively;][]{Davies-prep}, PerOB1
\protect\citep[a Galactic RSG cluster; cyan;][]{2014ApJ...788...58G} and NGC\,6822 (green).
Mean values for each data set are shown as enlarged points in the same style and colour.
The x-axis is reversed for comparison with Figure~\ref{fig:HRD}.\label{fig:TvsZ}
         }
\end{figure}

\begin{figure}
\includegraphics[width=9.0cm]{figures/N6822_HRD_all}
\caption{
H-R diagram for RSGs in PerOB1 (cyan), LMC (blue), SMC (red) and NGC\,6822 (green) which have stellar parameters obtained using the $J$-band method.
This figure shows that there appears to be no significant temperature differences between the four samples.
Solid grey lines show SMC-like metallicity evolutionary models including rotation
\protect\citep{2013A&A...558A.103G}.
Dashed grey lines show solar metallicity evolutionary models including rotation
\protect\citep{2012A&A...537A.146E}.\label{fig:HRD}
        }
\end{figure}


From these figures, we see no evidence for significant variations in the average temperatures of RSGs with respect to metallicity.
This is in contrast to current evolutionary models which display a change of $\sim$450K,
for a $M=15M_{\odot}$ model,
over a range of 0.7 dex~\citep{2012A&A...537A.146E,2013A&A...558A.103G}.

For solar metallicity, observations in PerOB1 are in good agreement with the models
\citep[see Figure 9 in][]{2014ApJ...788...58G}.
However, at SMC-like metallicity, the end-points of the models are systematically warmer than the observations.
The temperature of the end-points of the evolutionary models of massive stars could depend on the choice of the convective mixing-length parameter
\citep{1992A&AS...96..269S}.
That the models produce a higher temperature than observed could imply that the choice of a solar-like mixing-length parameter does not hold for higher-mass stars at lower metallicity.

There is evidence however,
that the average spectral type of RSGs tends towards an earlier spectral type with decreasing metallicity over this range
\citep{1979ApJ...231..384H,2012AJ....144....2L}.
We argue that these conclusions are not mutually exclusive.
Spectral types are derived for RSGs using the optical TiO band-heads at
$\sim$0.65$\mu$m,
whereas in this study temperatures are derived using near-IR atomic features
(as well as the line-free pseudo-continuum).
The strength of TiO bands are dependent upon metallicity which means that
the spectral classification for RSGs at a constant temperature will differ
\citep{2013ApJ...767....3D}.
Therefore, although historically spectral type has been used as a proxy for temperature, this assumption does not provide an accurate picture for RSGs.

% Recently,
% \cite{2013ApJ...767....3D} showed that the strength of TiO bands are dependent upon metallicity and that at lower metallicity, the TiO bands are significantly weaker.
% The trend in spectral type (or strength of the TiO absorption features) can be naturally explained by the decreasing abundance of the TiO molecule in lower metallicity environments.

% subsection temperatures_of_rsgs (end)
\subsection{AGB Contamination} % (fold)
\label{sub:AGB_contamination}
\textbf{
As mentioned in Section~\ref{sub:target_selection}, massive AGB stars are potential contaminants to our sample.
These stars have similar properties to RSGs and occupy similar mass ranges as lower-mass RSGs
\citep{2005ARA&A..43..435H},
however, their lifetimes are around $\sim$250\,Myr
\citep{2010MNRAS.401.1453D}.
\cite{1983ApJ...272...99W} argued for an AGB luminosity limit
(due to the limit on the mass of the degenerate core) of M$_{bol}\sim-7.1$.
Using this maximum luminosity, corrected for the distance to NGC\,6822,
yields $K$ = 14.0.
Four of our analysed stars have $K$-band magnitudes fainter than this limit,
but excluding the results for these does not significantly alter any of our conclusions
(and arguably, they would still be tracing the relatively young stellar population).
}

% subsection AGB_contamination (end)
% section discussion (end)

\section{Conclusions} % (fold)
\label{sec:conclusions}

KMOS spectroscopy of red supergiant stars (RSGs) in NGC\,6822 is presented.
The data were telluric corrected in two different ways and the standard 3-arm telluric method is shown to work as effectively (in most cases) as the more time expensive 24-arm telluric method.
Radial velocities of the targets are derived and are shown to be consistent with previous results in NGC\,6822, confirming their extragalactic nature.

Stellar parameters are calculated for 11 RSGs using the $J$-band analysis method outlined in
\cite{2010MNRAS.407.1203D}.
The average metallicity within NGC\,6822 is
$\bar{Z} = -0.52\pm 0.21$,
consistent with previous abundance studies of young stars.
We find an indication of a metallicity gradient within the central 1\,kpc,
however with a low significance caused by the small size and limited spatial extent of our RSG sample.
To conclusively assess the presence of a metallicity gradient among the young population within NGC\,6822 a larger systematic study of RSGs is needed.

The chemical abundances of the young and old stellar populations of NGC\,6822 are well explained by a simple closed-box chemical evolution model.
However, while an interesting result, we note that the closed-box model is unlikely to be a good assumption for this galaxy given its morphology.

The effective temperatures of RSGs in this study are compared to those of all RSGs analysed using the same method.
Using results which span 0.7 dex in metallicity (solar to SMC) within four galaxies, we find no evidence for a systematic variation in average effective temperature with respect to metallicity.
This is in contrast with evolutionary models for which a change in metallicity of 0.7 dex produces a shift in the temperature of RSGs of up to 450K.
We argue that an observed shift in average spectral type of RSGs over the same metallicity range (0.7 dex) does not imply a shift in average temperature.

These observations were taken as part of the KMOS Science Verification program.
With guaranteed time observations we have obtained data for RSGs in NGC\,300 and NGC\,55 at distances of $\sim$1.9\,Mpc,
as well as observations of super-star clusters in M\,83 and the Antennae galaxy at 4.5 and 20\,Mpc, respectively.
Owing to the fact that RSGs dominate the light output from super-star clusters
\citep{2013MNRAS.430L..35G} these clusters can be analysed in a similar manner
\citep{2014ApJ...787..142G},
which will provide metallicity measurements at distances a factor of 10 larger than using individual RSGs!
This work is the first step towards an ambitious general observation proposal to survey a large number of galaxies in the Local Volume,
motivated by the twin goals of investigating their abundance patterns,
while also calibrating the relationship between galaxy mass and metallicity in the Local Group.

% section conclusions (end)


%% If you wish to include an acknowledgments section in your paper,
%% separate it off from the body of the text using the \acknowledgments
%% command.


\acknowledgments

We thank the referee for a thorough review and constructive comments.
We thank Mike Irwin for providing the photometric catalogue from the WFCAM observations.
RPK and JZG acknowledge support by the National Science Foundation under grant AST-1108906.
NSO/Kitt Peak FTS data used here were produced by NSF/NOAO.

{\it Facilities:},
\facility{VLT (KMOS)}.

\begin{thebibliography}{}
\bibitem[Asplund et al.(2009)]{2009ARA&A..47..481A} Asplund, M., Grevesse, N., Sauval, A.~J., \& Scott, P.\ 2009, \araa, 47, 481

\bibitem[Battinelli et al.(2006)]{2006A&A...451...99B} Battinelli, P., Demers, S., \& Kunkel, W.~E.\ 2006, \aap, 451, 99

\bibitem[Bergemann et al.(2012)]{2012ApJ...751..156B} Bergemann, M.,
Kudritzki, R.-P., Plez, B., et al.\ 2012, \apj, 751, 156

\bibitem[Bergemann et al.(2013)]{2013ApJ...764..115B} Bergemann, M.,
Kudritzki, R.-P., W{\"u}rl, M., et al.\ 2013, \apj, 764, 115

\bibitem[Bovy
\& Rix(2013)]{2013ApJ...779..115B} Bovy, J., \& Rix, H.-W.\ 2013, \apj, 779, 115

\bibitem[Cannon et al.(2006)]{2006ApJ...652.1170C} Cannon, J.~M., Walter,
F., Armus, L., et al.\ 2006, \apj, 652, 1170

\bibitem[Cioni \& Habing(2005)]{2005A&A...429..837C} Cioni, M.-R.~L., \& Habing, H.~J.\ 2005, \aap, 429, 837

\bibitem[Cioni et
al.(2014)]{2014A&A...562A..32C} Cioni, M.-R.~L., Girardi, L., Moretti, M.~I., et al.\ 2014, \aap, 562, AA32

\bibitem[Clementini et al.(2003)]{2003ApJ...588L..85C} Clementini, G.,
Held, E.~V., Baldacci, L., \& Rizzi, L.\ 2003, \apjl, 588, L85

\bibitem[Davies et al.(2015)]{Davies-prep} Davies, B., Kudritzki,
R.-P., Gazak, J.~Z.,
et al., \apj, in press

\bibitem[Davies et al.(2010)]{2010MNRAS.407.1203D} Davies, B., Kudritzki,
R.-P., \& Figer, D.~F.\ 2010, \mnras, 407, 1203

\bibitem[Davies et al.(2013)]{2013ApJ...767....3D} Davies, B., Kudritzki,
R.-P., Plez, B., et al.\ 2013, \apj, 767, 3

\bibitem[Davies et
al.(2013)]{2013A&A...558A..56D} Davies, R.~I., Agudo Berbel, A., Wiezorrek, E., et al.\ 2013, \aap, 558, A56

\bibitem[de Blok
\& Walter(2000)]{2000ApJ...537L..95D} de Blok, W.~J.~G., \& Walter, F.\ 2000, \apjl, 537, L95

\bibitem[de Blok
\& Walter(2003)]{2003MNRAS.341L..39D} de Blok, W.~J.~G., \& Walter, F.\ 2003, \mnras, 341, L39

\bibitem[de Blok
\& Walter(2006)]{2006AJ....131..343D} de Blok, W.~J.~G., \& Walter, F.\ 2006, \aj, 131, 343

\bibitem[Demers et al.(2006)]{2006ApJ...636L..85D} Demers, S., Battinelli,
P., \& Kunkel, W.~E.\ 2006, \apjl, 636, L85

\bibitem[Doherty et al.(2010)]{2010MNRAS.401.1453D} Doherty, C.~L., Siess,
L., Lattanzio, J.~C., \& Gil-Pons, P.\ 2010, \mnras, 401, 1453

\bibitem[Ekstr{\"o}m et
al.(2012)]{2012A&A...537A.146E} Ekstr{\"o}m, S., Georgy, C., Eggenberger, P., et al.\ 2012, \aap, 537, A146

\bibitem[Evans et
al.(2011)]{2011A&A...527A..50E} Evans, C.~J., Davies, B., Kudritzki, R.-P., et al.\ 2011, \aap, 527, A50

\bibitem[Feast et al.(2012)]{2012MNRAS.421.2998F} Feast, M.~W., Whitelock,
P.~A., Menzies, J.~W., \& Matsunaga, N.\ 2012, \mnras, 421, 2998

\bibitem[Gazak et al.(2014)]{2014ApJ...788...58G} Gazak, J.~Z., Davies, B.,
Kudritzki, R., Bergemann, M., \& Plez, B.\ 2014, \apj, 788, 58

\bibitem[Gazak et al.(2014)]{2014ApJ...787..142G} Gazak, J.~Z., Davies, B.,
Bastian, N., et al.\ 2014, \apj, 787, 142

\bibitem[Gazak et al.(2013)]{2013MNRAS.430L..35G} Gazak, J.~Z., Bastian,
N., Kudritzki, R.-P., et al.\ 2013, \mnras, 430, L35

\bibitem[Georgy et
al.(2013)]{2013A&A...558A.103G} Georgy, C., Ekstr{\"o}m, S., Eggenberger, P., et al.\ 2013, \aap, 558, A103

\bibitem[Gratier et
al.(2010)]{2010A&A...512A..68G} Gratier, P., Braine, J., Rodriguez-Fernandez, N.~J., et al.\ 2010, \aap, 512, A68

\bibitem[Gustafsson et
al.(2008)]{2008A&A...486..951G} Gustafsson, B., Edvardsson, B., Eriksson, K., et al.\ 2008, \aap, 486, 951

\bibitem[Hern{\'a}ndez-Mart{\'{\i}}nez et
al.(2009)]{2009A&A...505.1027H} Hern{\'a}ndez-Mart{\'{\i}}nez, L., Pe{\~n}a, M., Carigi, L., \& Garc{\'{\i}}a-Rojas, J.\ 2009, \aap, 505, 1027

\bibitem[Herwig(2005)]{2005ARA&A..43..435H} Herwig, F.\ 2005, \araa, 43, 435

\bibitem[Houk
\& Smith-Moore(1988)]{1988mcts.book.....H} Houk, N., \& Smith-Moore, M.\ 1988, Michigan Catalogue of Two-dimensional Spectral Types for the HD Stars.~Volume 4, Declinations -26{$\deg$}.0 to -12{$\deg$}.0..Department of Astronomy, University of Michigan, Ann Arbor, MI 48109-1090, USA..,

\bibitem[Humphreys(1979)]{1979ApJ...231..384H} Humphreys, R.~M.\ 1979,
\apj, 231, 384

\bibitem[Huxor et al.(2013)]{2013MNRAS.429.1039H} Huxor, A.~P., Ferguson,
A.~M.~N., Veljanoski, J., Mackey, A.~D.,
\& Tanvir, N.~R.\ 2013, \mnras, 429, 1039

\bibitem[Hwang et al.(2011)]{2011ApJ...738...58H} Hwang, N., Lee, M.~G.,
Lee, J.~C., et al.\ 2011, \apj, 738, 58

\bibitem[Hwang et al.(2014)]{2014ApJ...783...49H} Hwang, N., Park, H.~S.,
Lee, M.~G., et al.\ 2014, \apj, 783, 49

\bibitem[Israel et
al.(1996)]{1996A&A...308..723I} Israel, F.~P., Bontekoe, T.~R., \& Kester, D.~J.~M.\ 1996, \aap, 308, 723

% \bibitem[Kausch et al.(submitted)]{Kausch-subm} Kausch, W.,
% et al., submitted.

\bibitem[Kirby et al.(2013)]{2013ApJ...779..102K} Kirby, E.~N., Cohen,
J.~G., Guhathakurta, P., et al.\ 2013, \apj, 779, 102

\bibitem[Komiyama et al.(2003)]{2003ApJ...590L..17K} Komiyama, Y., Okamura,
S., Yagi, M., et al.\ 2003, \apjl, 590, L17

\bibitem[Koribalski et al.(2004)]{2004AJ....128...16K} Koribalski, B.~S.,
Staveley-Smith, L., Kilborn, V.~A., et al.\ 2004, \aj, 128, 16


\bibitem[Lapenna et al.(2014)]{2014arXiv1410.5825L} Lapenna, E., Origlia,
L., Mucciarelli, A., et al.\ 2014, arXiv:1410.5825

\bibitem[Lee et al.(2006)]{2006ApJ...642..813L} Lee, H., Skillman, E.~D.,
\& Venn, K.~A.\ 2006, \apj, 642, 813

\bibitem[Letarte et al.(2002)]{2002AJ....123..832L} Letarte, B., Demers,
S., Battinelli, P., \& Kunkel, W.~E.\ 2002, \aj, 123, 832

\bibitem[Levesque
\& Massey(2012)]{2012AJ....144....2L} Levesque, E.~M., \& Massey, P.\ 2012, \aj, 144, 2

\bibitem[Massey(1998)]{1998ApJ...501..153M} Massey, P.\ 1998, \apj, 501,
153

\bibitem[Massey et al.(2007)]{2007AJ....134.2474M} Massey, P., McNeill,
R.~T., Olsen, K.~A.~G., et al.\ 2007, \aj, 134, 2474

\bibitem[McConnachie(2012)]{2012AJ....144....4M} McConnachie, A.~W.\ 2012,
\aj, 144, 4

\bibitem[Muschielok et
al.(1999)]{1999A&A...352L..40M} Muschielok, B., Kudritzki, R.~P., Appenzeller, I., et al.\ 1999, \aap, 352, L40

\bibitem[Nieva
\& Przybilla(2012)]{2012A&A...539A.143N} Nieva, M.-F., \& Przybilla, N.\ 2012, \aap, 539, A143

\bibitem[Nikolaev
\& Weinberg(2000)]{2000ApJ...542..804N} Nikolaev, S., \& Weinberg, M.~D.\ 2000, \apj, 542, 804

\bibitem[Pagel et al.(1980)]{1980MNRAS.193..219P} Pagel, B.~E.~J., Edmunds,
M.~G., \& Smith, G.\ 1980, \mnras, 193, 219

\bibitem[Pietrzy{\'n}ski et al.(2004)]{2004AJ....128.2815P}
Pietrzy{\'n}ski, G., Gieren, W., Udalski, A., et al.\ 2004, \aj, 128, 2815

\bibitem [Przybilla(2002)]{Przybilla02} Przybilla, N., 2002, Thesis, Munich University
Observatory, Ludwig-Maximilian-University Munich.

\bibitem[Richer
\& McCall(1995)]{1995ApJ...445..642R} Richer, M.~G., \& McCall, M.~L.\ 1995, \apj, 445, 642

\bibitem[Schaller et
al.(1992)]{1992A&AS...96..269S} Schaller, G., Schaerer, D., Meynet, G., \& Maeder, A.\ 1992, \aaps, 96, 269

\bibitem[Schlegel et al.(1998)]{1998ApJ...500..525S} Schlegel, D.~J.,
Finkbeiner, D.~P., \& Davis, M.\ 1998, \apj, 500, 525

\bibitem[Sharples et al.(2013)]{2013Msngr.151...21S} Sharples, R., Bender,
R., Agudo Berbel, A., et al.\ 2013, The Messenger, 151, 21

\bibitem[Sibbons et
al.(2012)]{2012A&A...540A.135S} Sibbons, L.~F., Ryan, S.~G., Cioni, M.-R.~L., Irwin, M., \& Napiwotzki, R.\ 2012, \aap, 540, A135


\bibitem[Tolstoy et al.(2001)]{2001MNRAS.327..918T} Tolstoy, E., Irwin,
M.~J., Cole, A.~A., et al.\ 2001, \mnras, 327, 918

\bibitem[Venn et al.(2001)]{2001ApJ...547..765V} Venn, K.~A., Lennon,
D.~J., Kaufer, A., et al.\ 2001, \apj, 547, 765

\bibitem[Wegner
\& Muschielok(2008)]{2008SPIE.7019E..0TW} Wegner, M., \& Muschielok, B.\ 2008, \procspie, 7019, 0T

\bibitem[Weisz et al.(2014)]{2014ApJ...789..147W} Weisz, D.~R., Dolphin,
A.~E., Skillman, E.~D., et al.\ 2014, \apj, 789, 147

\bibitem[Wolfire et al.(2003)]{2003ApJ...587..278W} Wolfire, M.~G., McKee,
C.~F., Hollenbach, D., \& Tielens, A.~G.~G.~M.\ 2003, \apj, 587, 278

\bibitem[Wood et al.(1983)]{1983ApJ...272...99W} Wood, P.~R., Bessell,
M.~S., \& Fox, M.~W.\ 1983, \apj, 272, 99

\end{thebibliography}

% \clearpage


%% The following command ends your manuscript. LaTeX will ignore any text
%% that appears after it.

\end{document}

%%
%% End of file
