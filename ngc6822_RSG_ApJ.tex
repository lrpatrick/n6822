%%
%% Beginning of file 'sample.tex'
%%
%% Modified 2005 December 5
%%
%% This is a sample manuscript marked up using the
%% AASTeX v5.x LaTeX 2e macros.

%% The first piece of markup in an AASTeX v5.x document
%% is the \documentclass command. LaTeX will ignore
%% any data that comes before this command.

%% The command below calls the preprint style
%% which will produce a one-column, single-spaced document.
%% Examples of commands for other substyles follow. Use
%% whichever is most appropriate for your purposes.
%%
%%\documentclass[12pt,preprint]{aastex}

%% manuscript produces a one-column, double-spaced document:

% \documentclass[manuscript]{aastex}
\documentclass[iop]{emulateapj}
\usepackage{natbib}
\citestyle{aa}
%% preprint2 produces a double-column, single-spaced document:

%% \documentclass[preprint2]{aastex}

%% Sometimes a paper's abstract is too long to fit on the
%% title page in preprint2 mode. When that is the case,
%% use the longabstract style option.

%% \documentclass[preprint2,longabstract]{aastex}

%% If you want to create your own macros, you can do so
%% using \newcommand. Your macros should appear before
%% the \begin{document} command.
%%
%% If you are submitting to a journal that translates manuscripts
%% into SGML, you need to follow certain guidelines when preparing
%% your macros. See the AASTeX v5.x Author Guide
%% for information.

\newcommand{\vdag}{(v)^\dagger}
\newcommand{\myemail}{lrp@roe.ac.uk}
\newcommand{\mdot}{\ensuremath{\dot{M}}}
\newcommand{\msun}{\ensuremath{M_{\odot}}}
\newcommand{\vsini}{\ensuremath{v_{\rm R} \sin i}}
\newcommand{\vrot}{\ensuremath{v_{\rm R}}}

\def\5{\footnotesize V\normalsize}
\def\4{\footnotesize IV\normalsize}
\def\3{\footnotesize III\normalsize}
\def\2{\footnotesize II\normalsize}
\def\1{\footnotesize I\normalsize}
\def\lam{$\lambda$}
\def\kms{$\mbox{km s}^{-1}$}
\def\p{$\phantom{:}$}
\def\a{$\phantom{^\ast}$}
\def\v{$\phantom{^{l}}$}
\def\pp{$\phantom{-}$}
\def\o{$\phantom{0}$}
\def\vr{$v_{\rm r}$}
%% You can insert a short comment on the title page using the command below.

% \slugcomment{Not to appear in Nonlearned J., 45.}

%% If you wish, you may supply running head information, although
%% this information may be modified by the editorial offices.
%% The left head contains a list of authors,
%% usually a maximum of three (otherwise use et al.).  The right
%% head is a modified title of up to roughly 44 characters.
%% Running heads will not print in the manuscript style.

\shorttitle{Red Supergiant Stars as Cosmic Abundance Probes}
\shortauthors{Patrick et al.}

%% This is the end of the preamble.  Indicate the beginning of the
%% paper itself with \begin{document}.

\begin{document}

%% LaTeX will automatically break titles if they run longer than
%% one line. However, you may use \\ to force a line break if
%% you desire.

\title{Red Supergiant Stars as Cosmic Abundance Probes: \\
    KMOS Observations in NGC\,6822}

%% Use \author, \affil, and the \and command to format
%% author and affiliation information.
%% Note that \email has replaced the old \authoremail command
%% from AASTeX v4.0. You can use \email to mark an email address
%% anywhere in the paper, not just in the front matter.
%% As in the title, use \\ to force line breaks.

\author{L. R. Patrick\altaffilmark{1},
C. J. Evans\altaffilmark{2,1},
B. Davies\altaffilmark{3},
R-P. Kudritzki\altaffilmark{4,5},
J. Z. Gazak\altaffilmark{4},
M. Bergemann\altaffilmark{6},
B. Plez\altaffilmark{7},
A. M. N. Ferguson\altaffilmark{1}}
% \affil{Institute for Astronomy, University of Edinburgh, Royal Observatory Edinburgh, Blackford Hill, Edinburgh EH9 3HJ}

% \author{C. D. Biemesderfer\altaffilmark{4,5}}
% \affil{National Optical Astronomy Observatories, Tucson, AZ 85719}
% \email{aastex-help@aas.org}

% \and

% \author{R. J. Hanisch\altaffilmark{5}}
% \affil{Space Telescope Science Institute, Baltimore, MD 21218}

%% Notice that each of these authors has alternate affiliations, which
%% are identified by the \altaffilmark after each name.  Specify alternate
%% affiliation information with \altaffiltext, with one command per each
%% affiliation.

\altaffiltext{1}{Institute for Astronomy, University of Edinburgh, Royal Observatory Edinburgh, Blackford Hill, Edinburgh EH9 3HJ, UK}
\altaffiltext{2}{UK Astronomy Technology Centre, Royal Observatory Edinburgh, Blackford Hill, Edinburgh EH9 3HJ, UK}
\altaffiltext{3}{Astrophysics Research Institute, Liverpool John Moores University, Liverpool Science Park ic2, 146 Brownlow Hill, Liverpool L3 5RF, UK}
\altaffiltext{4}{Institute for Astronomy, University of Hawaii, 2680 Woodlawn Drive, Honolulu, HI, 96822, USA}
\altaffiltext{5}{University Observatory Munich, Scheinerstr. 1, D-81679 Munich, Germany}
\altaffiltext{6}{Institute of Astronomy, University of Cambridge, Madingley Road, Cambridge CB3 0HA, UK}
\altaffiltext{7}{Laboratoire Univers et Particules de Montpellier, Universit\'e Montpellier 2, CNRS, F-34095 Montpellier, France}
%% Mark off your abstract in the ``abstract'' environment. In the manuscript
%% style, abstract will output a Received/Accepted line after the
%% title and affiliation information. No date will appear since the author
%% does not have this information. The dates will be filled in by the
%% editorial office after submission.

\begin{abstract}
We present near-IR spectroscopy of red supergiant (RSG) stars in NGC\,6822, obtained with the new VLT-KMOS instrument.
From comparisons with model spectra in the $J$-band we determine the metallicity of 11 confirmed RSGs, finding a mean value of [Z] $= -0.52 \pm 0.21$ which agrees well with previous abundance studies of young stars and H\2 regions.
We also find an indication for an abundance gradient within the central 1\,kpc, however with low significance because
of the small number of objects studied.
We compare our results those derived using older stellar populations and explain the discrepancy between the results using chemical evolution models.
By comparing the physical properties determined for RSGs in NGC\,6822 with those derived using the same technique in the Galaxy and the Magellanic Clouds we show that there appears to be no temperature evolution of RSGs with respect to metallicity, in contrast to what is seen with evolutionary models spanning the same rangesge in metallicity.
\end{abstract}

%% Keywords should appear after the \end{abstract} command. The uncommented
%% example has been keyed in ApJ style. See the instructions to authors
%% for the journal to which you are submitting your paper to determine
%% what keyword punctuation is appropriate.

\keywords{Galaxies: individual: NGC\,6822
-- stars: abundances
-- stars: supergiants}


\section{Introduction}

\label{sec:introduction}
A promising new method to directly probe chemical abundances in external galaxies is with $J$-band spectroscopy of red supergiant (RSG) stars.
With their peak flux at
$\sim$1\,$\mu$m and luminosities in excess of
10$^4$\,L$_\odot$, RSGs are extremely bright in the near-IR
(with $-$8\,$\le$\,M$_{J}$\,$\le$\,$-$11).
Therefore, RSGs are useful tools with which to map the chemical evolution of their host galaxies out to large distances.
To realise this goal
\cite{Davies10}, outlined a technique to derive metallicities of RSGs at moderate resolution
(R$\sim$3000) and a signal-to-noise ratio (S/N)
$\gtrsim$ 100.
This technique has recently been refined using observations of RSGs in the Magellanic Clouds
\citep{Davies14} and Perseus OB-1
\citep{2014ApJ...788...58G}.
Using absorption lines in the $J$-band from iron, silicon, titanium and eventually magnesium one can derive metallicity
([Z] = log (Z/Z$_{\odot}$)) as well as other stellar parameters
(effective temperature, surface gravity and microturbulence) by fitting model atmospheres to the observations.
% The exciting potential of this method compared to other tracers (e.g. blue supergiants, H\2 regions) is that it potentially provides direct abundance estimates of both iron and $\alpha$-elements (e.g., Si, Mg and Ti).
% This would provide information on the relative contributions from different nucleosynthesis channels to chemical enrichment of the host galaxy,
% i.e., the role of type II core-collapse vs. type Ia supernovae.
Owing to their intrinsic brightness,
RSGs are ideal candidates for studies of extragalactic environments in the near-IR.

To fully make use of the potential of RSGs for this science, multi-object spectrographs operating in the near-IR on 8-m class telescopes are essential.
These instruments allow us to observe a large sample of RSGs in a given galaxy, at a wavelength where RSGs are brightest.
In this context, the K-band Multi-Object Spectrograph
\citep[KMOS;][]{2013Msngr.151...21S} at the Very Large Telescope (VLT), Chile, is a powerful facility with which to explore our goals.
KMOS will enable determination of stellar abundances and radial velocities for RSGs towards distances of 10\,Mpc.
% Using KMOS, stellar abundances and radial velocities can be measured out towards 10\, Mpc, building a picture of abundance gradients and star formation histories in external galaxies over their entire spatial extent.
Further ahead, a near-IR multi-object spectrograph on a 40-m class telescope, combined with the excellent image quality from adaptive optics,
will enable abundance estimates for individual stars in galaxies out to tens of Mpc,
a significant volume of the local universe containing entire galaxy clusters
\citep{Evans11}.

Here we present KMOS observations of RSGs in the dwarf irregular galaxy NGC\,6822,
at a distance of $\sim$0.5\,Mpc
\cite[see e.g.][]{2003ApJ...588L..85C,2004AJ....128.2815P,2005A&A...429..837C,2012MNRAS.421.2998F}.
Its present-day iron abundance is thought to be intermediate to that of the LMC and SMC
\citep{2013ApJ...779..102K},
% 1999A&A...352L..40M,2006ApJ...647..970L},
but we lack firmer constraints on both its metallicity and its recent chemical evolution.
Observations of two A-type supergiants by
\cite{2001ApJ...547..765V} provided a first estimate of stellar abundances
log (Fe/H) $+12=7.01\pm0.22$
log(O/H)   $+12=8.36\pm0.19$, based on LTE line formation calculations for these elements.
A detailed NLTE study
\citep{Przybilla} for one of these objects confirmed these results with
$6.96\pm 0.09$ for iron and $8.30\pm0.02$ for oxygen.
Compared to the solar abundances of 7.50 and 8.69 of these elements
\citep{2009ARA&A..47..481A} this indicates about 0.5dex lower in NGC\,6822.
A study of oxygen abundances in HII-regions by 
\citep{2006ApJ...642..813L} finds a value of $8.11\pm0.1$ confirming the low metallicity.


NGC\, 6822 is a relatively isolated Local Group galaxy, which does not seem to be associated with either M31 or the Milky Way.
It appears to have a relatively large extended stellar halo
\citep{2002AJ....123..832L,2014ApJ...783...49H} 
as well as an extended H\1 disk containing tidal arms and a possible H\1 companion
\citep{2000ApJ...537L..95D}.
The H\1 disk is orientated perpendicular to the distribution of old halo stars and has an associated population of blue stars 
\citep{2003MNRAS.341L..39D,2003ApJ...590L..17K}.
This led \cite{2006ApJ...636L..85D} to label the system as a
~\textquoteleft
polar ring galaxy\textquoteright.
A population of remote star clusters aligned with the elongated old stellar halo have been recently discovered
\citep{2011ApJ...738...58H,2013MNRAS.429.1039H}.
The extended structures of NGC\,6822 suggest some form of recent interaction.

In addition, there is evidence for a relatively constant star-formation history within the central 5\,kpc
\citep{2014ApJ...789..147W}
with multiple stellar populations
\citep{2006A&A...451...99B,2012A&A...540A.135S}.
This includes evidence for recent star formation in the form of a known population of massive stars as well as a number of H\2 regions
\citep{2001ApJ...547..765V,2006AJ....131..343D,2009A&A...505.1027H,Levesque12}.

In this paper we present near-IR spectroscopy of RSGs in NGC\,6822 from KMOS.
In Section~\ref{sec:observations} we describe the observations.
Section~\ref{sec:data_reduction_and_analysis} describes the data reduction and tests performed for these observations, 
radial velocities are computed in Section~\ref{sec:membership},
Section~\ref{sec:results} details the derived stellar parameters and quantifies the abundance gradient in NGC\,6822, in Section~\ref{sec:discussion} we discuss our results and in
Section~\ref{sec:conclusions} we conclude the paper.

% section introduction (end)

\section{Observations}

%% In a manner similar to \objectname authors can provide links to dataset
%% hosted at participating data centers via the \dataset{} command.  The
%% second curly bracket argument is printed in the text while the first
%% parentheses argument serves as the valid data set identifier.  Large
%% lists of data set are best provided in a table (see Table 3 for an example).
%% Valid data set identifiers should be obtained from the data center that
%% is currently hosting the data.
%%
%% Note that AASTeX interprets everything between the curly braces in the
%% macro as regular text, so any special characters, e.g. "#" or "_," must be
%% preceded by a backslash. Otherwise, you will get a LaTeX error when you
%% compile your manuscript.  Special characters do not
%% need to be escaped in the optional, square-bracket argument.

\label{sec:observations}

% section observations (end)
\subsection{Target Selection} % (fold)
\label{sub:target_selection}

Our targets were selected from optical photometry
\citep{Massey07}, combined with near-IR ($JHK{_s}$) photometry
\cite[for details see][]{2012A&A...540A.135S} from the Wide-Field Camera (WFCAM) on the United Kingdom Infra-Red Telescope (UKIRT).
The two catalogues were cross-matched and only sources classified as stellar in the photometric catalogues for all filters were considered.

Our spectroscopic targets were selected principally based on their optical colours, as defined by
\cite{Massey98} and
\cite{Levesque12}.
Figure~\ref{fig:BVR} shows cross-matched stars, with the dividing line at $(B - V) = 1.25 \times (V - R)+0.45$.
All stars redder than this line, with $V - R > 0.6$, are potential RSGs.

To increase confidence in our targets the targets selected via their optical colours are also required to meet an additional criteria based on near-IR photometry, using the known location of RSGs in the $J-K$ colour-magnitude diagram
\citep{Nikolaev00}, as shown in Figure~\ref{fig:JK}.

The combined selection methods yielded 58 targets, from which 19 stars were selected to observe with KMOS, as shown in Figure
\ref{fig:N6822}.
The selection of the final targets was defined by the KMOS arm allocation software {\sc KARMA} with a prior for the brightest targets.
Of the 19 candidates, eight were previously spectroscopically confirmed as RSGs by
\cite{Levesque12}.


\begin{figure}
 \includegraphics[width=9.0cm]{figures/n6822_bvr_paper}
 \caption{
          Two-colour diagram for stars with good detections in optical and near-IR bands in NGC\,6822.
          The black dashed line marks the selection criteria using optical colours, as defined by~\protect\cite{Levesque12}.
          Red circles mark all stars which satisfy the selection criteria.
           % and our $J$-$K$, $K$ cut.
          Large blue stars denote targets observed with KMOS.
          Solid black line marks the reddening vector
          \protect\citep{1998ApJ...500..525S}.
         }
 \label{fig:BVR}
\end{figure}

\begin{figure}
 \includegraphics[width=9.0cm]{figures/n6822_jk_paper}
 \caption{
          Near-IR colour-magnitude diagram (CMD) for stars classified as stellar sources in both catalogues, plotted using the same symbols as Figure~\ref{fig:BVR}.
          This CMD is used to aid the optical selection criterion.
          Solid black line marks the reddening vector
          \protect\citep{1998ApJ...500..525S}.
         }
 \label{fig:JK}
\end{figure}


\begin{figure}
 \includegraphics[width=9.0cm]{figures/N6822_RSGs_paper}
 \caption{Spatial extent of the KMOS targets over a Digital Sky Survey (DSS) image of NGC\,6822.
          Blue filled circles indicate the locations of the observed red supergiant stars.
          Red circles indicate the positions of red supergiant candidates selected using a two-colour selection method~\citep{Levesque12}.
          % To add: North East points, 1\,kpc scale
          % \textbf{This figure is useless in black and white!}
          }
 \label{fig:N6822}
\end{figure}

% section target_selection (end)

\subsection{KMOS Observations} % (fold)
\label{sub:observations}

The observations were obtained as part of the KMOS Science Verification (SV) program on 30 June 2013 (PI: Evans 60.A-9452(A)),
with a total exposure time of 2400\,s
(comprising 8\,$\times$\,300\,s detector integrations).
% Observations for this study are from the new KMOS instrument on the VLT, Chile.
KMOS has 24 deployable integral-field units (IFU) each of which covers an area of
2\farcs8 $\times$ 2\farcs8 over a 7\farcm2 field-of-view.
The 24 IFUs are split into three groups of eight and each group is relayed to different spectrographs with separate detectors.

Offset sky frames
(0\farcm5 to the east) were interleaved between the science observations in an object (O), sky (S) sequence of:
O,\,S,\,O,\,O.
The observations were performed with the $YJ$ grating
(giving coverage from 1.0 to 1.359$\mu$m);
estimates of the delivered resolving power for each spectrograph (obtained from the KMOS/esorex pipeline for two arc lines) are listed in Table~\ref{tb:res}.

In addition to the science observations a standard set of KMOS calibration frames were obtained consisting of dark, flat and arc-lamp calibrations (with flats and arcs taken at six different rotator angles).
A telluric standard star was observed with the arms configured in the science positions, i.e. using the {\em KMOS\_spec\_cal\_stdstarscipatt} template in which the standard star is observed sequentially through all the IFUs.
The observed standard was HIP97618, with a spectral type of B6\,III
\citep{1988mcts.book.....H}.
 % and a telluric standard star observed in each KMOS IFU.
% The telluric standard was observed with the arms configured in the science positions, i.e. using the {\em KMOS\_spec\_cal\_stdstarscipatt} template in which the standard star is observed sequentially through all the IFUs.

The {\sc karma} configuration software
\citep{2008SPIE.7019E..27W} was used to allocate the KMOS arms to 19 RSG science targets.
A summary of the observed targets is given in
Table~\ref{tb:obs-params}.

To perform the analysis to a satisfactory standard,
the ideal S/N per resolution element for any given spectra must be $\gtrsim$ 100
\citep{2014ApJ...788...58G}.
We estimated the observed S/N and find that we achieve the required S/N for (nearly) all targets.


% \footnotetext{http://www.usm.uni-muenchen.de/people/wegner/kmos/en/karma.php}

\begin{table*}
\caption{Measured velocity resolution and resolving power across each detector.\label{tb:res}}
\scriptsize
\begin{center}
\begin{tabular}{crcccc}
\hline
\hline
Det. & IFUs & \multicolumn{2}{c}{Ar\,\lam1.12430\,$\mu$m}
            & \multicolumn{2}{c}{Ne\,\lam1.17700\,$\mu$m} \\
 & & FWHM [\kms] & $R$ & FWHM [\kms] & $R$ \\
  \hline
1 & 1-8 & \o85.45\,$\pm$\,2.67 & 3\,511$\pm$\,110 &
          \a88.04\,$\pm$\,2.67 & 3\,408$\pm$\,103 \\
2 & 9-16 & \o80.30\,$\pm$\,3.05 & 3\,736$\pm$\,142 &
          \a82.83\,$\pm$\,2.48 & 3\,622$\pm$\,108 \\
3 & 17-24 & 101.25\,$\pm$\,2.99 & 2\,963$\pm$\,87\a &
            103.23\,$\pm$\,2.73 & 2\,906$\pm$\,77\a \\
\hline
\end{tabular}
\end{center}
\end{table*}



% % \clearpage
% \begin{deluxetable}{crcccc}
% \tabletypesize{\scriptsize}
% % \rotate
% \tablecaption{Full-width half-maximum (FWHM) velocity resolution and equivalent
% resolving power ($R$) determined by the KMOS pipeline for two diagnostic arc-lines for the $YJ$
% grating, for each of the three detectors.\label{tb:res}}
% \tablewidth{0pt}
% \tablehead{
% \colhead{Det.} & \colhead{IFUs} &
% \multicolumn{2}{c}{Ar\,\lam1.12430\,$\mu$m} & \multicolumn{2}{c}{Ne\,\lam1.17700\,$\mu$m} \\
% & & \colhead{FWHM [\kms]} & \colhead{$R$} & \colhead{FWHM [\kms]} & \colhead{$R$}
% }
% \startdata
% 1 & 1-8 & \o85.45\,$\pm$\,2.67 & 3\,511$\pm$\,110 &
%           \a88.04\,$\pm$\,2.67 & 3\,408$\pm$\,103 \\
% 2 & 9-16 & \o80.30\,$\pm$\,3.05 & 3\,736$\pm$\,142 &
%           \a82.83\,$\pm$\,2.48 & 3\,622$\pm$\,108 \\
% 3 & 17-24 & 101.25\,$\pm$\,2.99 & 2\,963$\pm$\,87\a &
%             103.23\,$\pm$\,2.73 & 2\,906$\pm$\,77\a \\
% \enddata
% \end{deluxetable}



\begin{table*}
\caption{
        Summary of VLT-KMOS targets in NGC\,6822.\label{tb:obs-params}
        }
\scriptsize
\begin{center}
\begin{tabular}{llrccccccccl}
 \hline
 \hline
Name & Alt. name & S/N & $\alpha$ (J2000) & $\delta$ (J2000) & $B$ & $V$ & $R$ & $J$ & $H$ & $K_{\rm s}$ & Notes \\
 \hline
J194443.81$-$144610.7  &  RSG5   & 222.8 &   19:44:43.81  &  $-$14:46:10.7  &  20.83  &  18.59  &  17.23  &  14.16  &  13.37  &  13.09 & Sample\\
J194445.98$-$145102.4  &  RSG8   & 119.6 &   19:44:45.98  &  $-$14:51:02.4  &  20.91  &  18.96  &  17.89  &  15.53  &  14.72  &  14.52 & Sample\\
J194447.13$-$144627.1  &  RSG9   &  94.2 &   19:44:47.13  &  $-$14:46:27.1  &  21.30  &  19.41  &  18.41  &  16.13  &  15.35  &  15.12 \\
J194447.81$-$145052.5  &  RSG12  & 211.4 &   19:44:47.81  &  $-$14:50:52.5  &  20.74  &  18.51  &  17.22  &  14.37  &  13.58  &  13.30 & LM12, Sample \\
J194450.54$-$144801.6  &  RSG16  & 104.2 &   19:44:50.54  &  $-$14:48:01.6  &  20.83  &  18.95  &  17.97  &  15.75  &  14.98  &  14.79 \\
J194451.64$-$144858.0  &  RSG21  & 105.4 &   19:44:51.64  &  $-$14:48:58.0  &  21.33  &  19.45  &  18.32  &  15.81  &  14.95  &  14.72 \\
J194453.46$-$144552.6  &  RSG24  & 144.6 &   19:44:53.46  &  $-$14:45:52.6  &  20.36  &  18.43  &  17.38  &  15.06  &  14.30  &  14.08 & LM12, Sample \\
J194453.46$-$144540.1  &  RSG25  & 103.3 &   19:44:53.46  &  $-$14:45:40.1  &  20.88  &  19.14  &  18.17  &  15.95  &  15.16  &  14.98 & LM12, Sample \\
J194454.46$-$144806.2  &  RSG29  & 201.0 &   19:44:54.46  &  $-$14:48:06.2  &  20.56  &  18.56  &  17.35  &  14.43  &  13.67  &  13.34 & LM12, Sample\\
J194454.54$-$145127.1  &  RSG30  & 301.5 &   19:44:54.54  &  $-$14:51:27.1  &  19.29  &  17.05  &  15.86  &  13.43  &  12.66  &  12.42 & LM12, Sample \\
J194455.70$-$145155.4  &  RSG34  & 326.7 &   19:44:55.70  &  $-$14:51:55.4  &  19.11  &  16.91  &  15.74  &  13.43  &  12.70  &  12.43 & LM12, Sample \\
J194455.93$-$144719.6  &  RSG36  &  99.6 &   19:44:55.93  &  $-$14:47:19.6  &  21.43  &  19.56  &  18.52  &  16.14  &  15.33  &  15.14 & LM12 \\
J194456.86$-$144858.5  &  RSG39  & 106.1 &   19:44:56.86  &  $-$14:48:58.5  &  21.05  &  19.06  &  18.04  &  15.81  &  15.05  &  14.85 \\
J194457.31$-$144920.2  &  RSG40  & 283.5 &   19:44:57.31  &  $-$14:49:20.2  &  19.69  &  17.41  &  16.20  &  13.52  &  12.76  &  12.52 & LM12, Sample \\
J194459.14$-$144723.9  &  RSG45  & 123.6 &   19:44:59.14  &  $-$14:47:23.9  &  21.30  &  19.17  &  18.05  &  15.58  &  14.74  &  14.50 \\
J194500.24$-$144758.9  &  RSG47  & 109.6 &   19:45:00.24  &  $-$14:47:58.9  &  21.27  &  19.20  &  18.10  &  15.60  &  14.80  &  14.57 \\
J194500.53$-$144826.5  &  RSG49  & 167.4 &   19:45:00.53  &  $-$14:48:26.5  &  20.84  &  18.75  &  17.51  &  14.70  &  13.86  &  13.61 & Sample\\
J194506.98$-$145031.1  &  RSG55  & 104.0 &   19:45:06.98  &  $-$14:50:31.1  &  21.06  &  19.12  &  18.06  &  15.74  &  14.94  &  14.78 & Sample\\

\hline
\end{tabular}
\end{center}
\tablecomments{Optical data from
\protect\cite{Massey07}, near-IR data from the UKIRT survey
\protect\cite[see][for details]{2012A&A...540A.135S}.
Radial-velocity estimates are described in
Section~\ref{sec:membership}.
Targets with the comment \textquoteleft
LM12\textquoteright~are those observed by
\protect\cite{Levesque12}.
Targets with the comment \textquoteleft Sample\textquoteright
~are those used for analysis in this paper.}
\end{table*}


% \clearpage
% \begin{deluxetable}{llrccccccccl}
% \tabletypesize{\scriptsize}
% % \rotate
% \tablecaption{Summary of VLT-KMOS targets in NGC\,6822. Optical data from
% \protect\cite{Massey07}, near-IR data from the UKIRT survey
% \protect\cite[see][for details]{2012A&A...540A.135S}.
% Radial-velocity estimates are described in
% Section~\ref{sec:membership}.
% Targets with the comment \textquoteleft
% LM12\textquoteright~are those observed by
% \protect\cite{Levesque12}.
% Targets with the comment \textquoteleft Sample\textquoteright
% ~are those used for analysis in this paper.\label{tb:obs-params}}
% \tablewidth{0pt}
% \tablehead{
% \colhead{Name} & \colhead{Alt. name} & \colhead{S/N} & \colhead{$\alpha$ (J2000)} & \colhead{$\delta$ (J2000)} &
% \colhead{$B$} & \colhead{$V$} & \colhead{$R$} & \colhead{$J$} & \colhead{$H$} & \colhead{$K$} & \colhead{Notes}
% }
% \startdata
% J194443.81$-$144610.7  &  RSG5   & 222.8 &   19:44:43.81  &  $-$14:46:10.7  &  20.83  &  18.59  &  17.23  &  14.16  &  13.37  &  13.09 & Sample\\
% J194445.98$-$145102.4  &  RSG8   & 119.6 &   19:44:45.98  &  $-$14:51:02.4  &  20.91  &  18.96  &  17.89  &  15.53  &  14.72  &  14.52 & Sample\\
% J194447.13$-$144627.1  &  RSG9   &  94.2 &   19:44:47.13  &  $-$14:46:27.1  &  21.30  &  19.41  &  18.41  &  16.13  &  15.35  &  15.12 \\
% J194447.81$-$145052.5  &  RSG12  & 211.4 &   19:44:47.81  &  $-$14:50:52.5  &  20.74  &  18.51  &  17.22  &  14.37  &  13.58  &  13.30 & LM12, Sample \\
% J194450.54$-$144801.6  &  RSG16  & 104.2 &   19:44:50.54  &  $-$14:48:01.6  &  20.83  &  18.95  &  17.97  &  15.75  &  14.98  &  14.79 \\
% J194451.64$-$144858.0  &  RSG21  & 105.4 &   19:44:51.64  &  $-$14:48:58.0  &  21.33  &  19.45  &  18.32  &  15.81  &  14.95  &  14.72 \\
% J194453.46$-$144552.6  &  RSG24  & 144.6 &   19:44:53.46  &  $-$14:45:52.6  &  20.36  &  18.43  &  17.38  &  15.06  &  14.30  &  14.08 & LM12, Sample \\
% J194453.46$-$144540.1  &  RSG25  & 103.3 &   19:44:53.46  &  $-$14:45:40.1  &  20.88  &  19.14  &  18.17  &  15.95  &  15.16  &  14.98 & LM12, Sample \\
% J194454.46$-$144806.2  &  RSG29  & 201.0 &   19:44:54.46  &  $-$14:48:06.2  &  20.56  &  18.56  &  17.35  &  14.43  &  13.67  &  13.34 & LM12, Sample\\
% J194454.54$-$145127.1  &  RSG30  & 301.5 &   19:44:54.54  &  $-$14:51:27.1  &  19.29  &  17.05  &  15.86  &  13.43  &  12.66  &  12.42 & LM12, Sample \\
% J194455.70$-$145155.4  &  RSG34  & 326.7 &   19:44:55.70  &  $-$14:51:55.4  &  19.11  &  16.91  &  15.74  &  13.43  &  12.70  &  12.43 & LM12, Sample \\
% J194455.93$-$144719.6  &  RSG36  &  99.6 &   19:44:55.93  &  $-$14:47:19.6  &  21.43  &  19.56  &  18.52  &  16.14  &  15.33  &  15.14 & LM12 \\
% J194456.86$-$144858.5  &  RSG39  & 106.1 &   19:44:56.86  &  $-$14:48:58.5  &  21.05  &  19.06  &  18.04  &  15.81  &  15.05  &  14.85 \\
% J194457.31$-$144920.2  &  RSG40  & 283.5 &   19:44:57.31  &  $-$14:49:20.2  &  19.69  &  17.41  &  16.20  &  13.52  &  12.76  &  12.52 & LM12, Sample \\
% J194459.14$-$144723.9  &  RSG45  & 123.6 &   19:44:59.14  &  $-$14:47:23.9  &  21.30  &  19.17  &  18.05  &  15.58  &  14.74  &  14.50 \\
% J194500.24$-$144758.9  &  RSG47  & 109.6 &   19:45:00.24  &  $-$14:47:58.9  &  21.27  &  19.20  &  18.10  &  15.60  &  14.80  &  14.57 \\
% J194500.53$-$144826.5  &  RSG49  & 167.4 &   19:45:00.53  &  $-$14:48:26.5  &  20.84  &  18.75  &  17.51  &  14.70  &  13.86  &  13.61 & Sample\\
% J194506.98$-$145031.1  &  RSG55  & 104.0 &   19:45:06.98  &  $-$14:50:31.1  &  21.06  &  19.12  &  18.06  &  15.74  &  14.94  &  14.78 & Sample\\
% \enddata
% \end{deluxetable}



% subsection KMOS observations (end)

\section{Data Reduction and Analysis} % (fold)
\label{sec:data_reduction_and_analysis}

The observations were reduced using the recipes provided by the Software Package for Astronomical Reduction with KMOS
\citep[SPARK;][]{2013A&A...558A..56D}.
The standard KMOS/esorex routines were used to calibrate and reconstruct the science and standard-star data cubes as outlined by
\cite{2013A&A...558A..56D}.
% On-sky flatfield exposures taken on the previous night were used to implement a more precise flatfield correction.
Sky subtraction was performed using the standard KMOS recipes and telluric correction performed using two different strategies.
% and an additional second-order sky subtraction technique is implemented~\citep{2007MNRAS.375.1099D}.
Throughout the analysis presented in this section all the spectra have been extracted from their respective data cubes using a consistent method (i.e. the optimal extraction method within the pipeline).


\subsection{Three-arm vs 24-arm Telluric Correction} % (fold)
\label{sub:three_arm_vs_24_arm_telluric_correction}

The standard template for telluric observations with KMOS is to observe a standard star in one IFU in each of the three KMOS spectrographs.
However, there is an alternative template which allows users to observe a standard in each of the 24 IFUs.
This strategy should provide an optimum telluric correction for the KMOS IFUs but reduces observing efficiency.

A comparison between the two methods in the H-band was given by
\cite{2013A&A...558A..56D},
who concluded that using the more efficient three-arm method was suitable for most science purposes.
However, an equivalent analysis in the YJ-band was not available.
To determine if the more rigorous telluric routine is required for our analysis,
we observed a telluric standard star (HIP97618) in each of the 24 IFUs.
This gave us the tools to investigate both telluric correction methods on one data set,
and to directly compare the two results.

We compared the telluric spectrum in each IFU with that of the IFU used by the three-arm template in each of the three KMOS spectrographs.
Figure~\ref{fig:IFU_compare} shows the differences between the telluric spectra across the IFUs,
where the differences between the IFUs in the YJ-band are comparable to those in the H-band
\cite[cf. Fig.7 from][]{2013A&A...558A..56D}.
The agreement between the IFUs in our region of interest (1.15 $-$ 1.22$\mu$m) is generally very good.


\begin{figure*}
 \begin{center}
 \includegraphics[width=12.0cm]{figures/N6822_t_compare}
 \caption{
    Comparison of J-band spectra of the same standard star in each IFU.
    The ratio of each spectrum compared to that from the IFU used in the three-arm telluric method is shown,
    with their respective mean and standard deviation ($\mu$ and $\sigma$).
    Red lines indicate $\mu$~=~1.0, $\sigma$~=~0.0 for each ratio.
    Blue shaded area signifies region where the J-band analysis fitting takes place.
    Within this region, in general, the discrepancies between the IFUs is small.
    This is reflected in the standard deviation values when only considering this region.
    IFUs 13 and 16 are omitted as no data was taken with these IFUs. \label{fig:IFU_compare}
          }
 \end{center}
\end{figure*}

To quantify the difference the two telluric methods would make to our analysis,
we performed the steps described in
Section~\ref{sub:ngc6822_telluric_correction} for both templates.
We then used both sets of reduced science data
(reduced with both methods of the telluric correction) to compute stellar parameters for our targets.
The results of this comparison are detailed in Section
\ref{sub:telluric_comparison}.

% subsection three_arm_vs_24_arm_telluric_correction (end)

\subsection{Telluric Correction Implementation} % (fold)
\label{sub:ngc6822_telluric_correction}

To improve the accuracy of the telluric correction,
for both methods mentioned above,
we implemented some additional recipes beyond those of the KMOS/esorex pipeline.
These recipes are employed to account for two different effects which could potentially degrade the quality of the telluric correction.
The first corrects for any potential shift in wavelength between each science spectrum and its associated telluric spectrum.
The most effective way to implement this correction is to cross-correlate each pair of science and telluric spectra.
% Performing a cross-correlation between two spectra with similar spectral features allows one to identify a shift in wavelength space as a maximum in the cross-correlation will be produced when the two spectra are exactly aligned.
Any shift between the two spectra is then applied to the initial telluric spectrum using a cubic-spline interpolation routine.

The second correction applied is a simple spectral scaling algorithm.
This routine corrects for differences in line intensity of the most prominent features common to both the telluric and science spectra.
To find the optimal scaling parameter the following formula is used,

\begin{equation} \label{eq:shiftandres}
T_{2} = (T_{1} + c) / (T_{1} - c),
\end{equation}

\noindent where $T_{2}$ is the corrected telluric spectrum, $T_{1}$ is the initial telluric spectrum and $c$ is the scaling parameter.

To determine the required scaling,
telluric spectra are computed for $-0.5 < c < 0.5$ in increments of 0.02
(where a perfect value, i.e. no difference in line strength would be $c = 0$).
The standard deviation is then computed for each rtelluric-corrected science spectrum, and the minimum value of the standard-deviation matrix defines the optimum scaling.
For this algorithm, only the region of interest for our analysis is considered (i.e. $1.15 - 1.22\mu$m).

The final set of telluric spectra,
generated using the KMOS/esorex routines and modified using the additional routines described above,
are used to correct their respective science observations for the effects of the Earth's atmosphere.

% subsection ngc6822_telluric_correction (end)


\begin{figure*}
 %\vspace{302pt}
 \begin{center}
\includegraphics[width=15.cm]{figures/N6822_mod_fit.pdf}
\caption{KMOS Spectra of the NGC\,6822 RSGs and their associated best-fit model spectra
(black and red lines respectively).
Sine of the main diagnostic lines are marked above.
         }
\label{fig:model_fits}
\end{center}
\end{figure*}


% section data_reduction_and_analysis (end)


\section{NGC\,6822 Membership} % (fold)
\label{sec:membership}
Candidates were selected using their optical and near-IR colours,
however, these criteria alone are not enough to conclusively determine whether these stars are genuine NGC\,6822 members.
Foreground Galactic stars with similar spectral types (i.e. M-dwarfs) could potentially contaminate our sample and bias our conclusions.
NGC\,6822 is at low galactic latitude (b$\sim -18$) which makes Galactic foreground contamination particularly important.

To determine membership of NGC\,6822 we estimate stellar radial velocities from the KMOS spectra for each target.
Assigning membership to NGC\,6822 also confirms the luminosity class of our targets as supergiants.
Relative radial velocities are estimated by cross-correlating the spectrum of each source,
within the 1.15-1.23\,$\mu$m region,
against a reference spectrum,
chosen to be the highest S/N spectrum in our sample, RSG34.
Figure
\ref{fig:Vrad} shows the relative radial velocity for each target with respect to the reference target.
RSG5 has a velocity which is just below three-sigma from the median
and, as such, may not be associated with the same kinematic body as the other objects.
Absolute radial velocities are not calculated here as the absolute wavelength calibration of the data is uncertain.

% Figures & Tables:

\begin{figure}
 \includegraphics[width=9.0cm]{figures/N6822_RvsVrad}
 \caption{
 Relative radial velocities of KMOS targets as a function of radial distance from the centre of NGC\,6822.
 Radial velocity estimates are performed by cross-correlating each target with a reference spectrum (RSG34).
 The median of the distribution is marked by the red line.
 One median absolute deviation is marked by the green lines.
 The potential outlier at $\sim-35$kms$^{-1}$(RSG5) is more than three-sigma from the median.
 }
 \label{fig:Vrad}
\end{figure}

% subsection ngc6822_membership (end)

\section{Results} % (fold)
\label{sec:results}
Initial inspection of the spectra revealed minor residuals from the sky subtraction process.
Any residual sky subtraction features could potentially influence our results by perturbing the continuum placement within the model fits, which is an important aspect of the fitting process
\citep[see][for more discussion]{Davies14,2014ApJ...788...58G}.
Thus, pending a more rigorous treatment of the data
(e.g. to take into account the changing spectral resolution across the array),
we exclude objects showing residual sky-subtraction features from our analysis.
Of the 18 observed targets, 11 were used to derive stellar parameters
as indicated in Table~\ref{tb:obs-params}.

Stellar parameters
(metallicity, effective temperature, surface gravity and microturbulence)
have been derived using the J-band analysis technique described by
\cite{Davies10} and demonstrated by
\cite{Davies14} and
\cite{2014ApJ...788...58G}.
To determine physical parameters, this technique uses a grid of model atmospheres to fit the observational data.
The resolution of the models is determined by the measured resolution from the KMOS/esorex pipeline (Table~\ref{tb:res}).
Model atmospheres were generated using the MARCS code
\citep{Gustafsson08} where the range of parameters is defined in
Table~\ref{tb:mod_range}.
The precision of the models is increased by including departures from local thermodynamic equilibrium (LTE) in some of the strongest Fe, Ti and Si atomic lines
\citep{2012ApJ...751..156B,2013ApJ...764..115B}.
The two strong magnesium lines are excluded from the analysis as these lines are known to be affected strongly by non-LTE effects
(see Figure~\ref{fig:model_fits} where the two MgI lines are systematically under- and over-estimated respectively).
The non-LTE line formation of Mg I lines will be explored in an upcoming paper.

\begin{table}
\caption{
Model grid ranges used for analysis.\label{tb:mod_range}
         }
\scriptsize
\begin{center}
\begin{tabular}{lccc}
 \hline
 \hline
  Model Parameter & Min. & Max. & Step size \\
 \hline
T$_{fit}$ (K)        & 3400 & 4000 & 100 \\
                     & 4000 & 4400 & 200 \\
$[$Z$]$ (dex)   & $-$1.50 & 1.00  & 0.25\\
log $g$ (cgs)  & $-$1.0\o & 1.0\o & 0.5\o \\
 $\xi$ (\kms)  & \pp1.0\o & 6.0\o & 1.0\o\\
 \hline
\end{tabular}
\end{center}
\end{table}


% subsection telluric_comparison (end)

\subsection{Telluric Comparison} % (fold)
\label{sub:telluric_comparison}

We used these SV data to determine which of the two telluric standard star routines is most appropriate for our analysis.
By means of comparison, we reduced the science data using both telluric correction routines and, using our RSG model grid, computed stellar parameters.
Table~\ref{tb:stellar-params} details the stellar parameters derived for each target using both telluric methods, with the derived stellar parameters compared in
Figure~\ref{fig:3vs24AT}.
% We compare the stellar parameters derived using the two telluric correction methods in Figure~\ref{fig:3vs24AT}.
The mean difference between the metallicity values for two methods is
$\Delta [Z] = 0.01\pm 0.10$.
Therefore, for our analysis, there is no significant difference between the two telluric methods.


\begin{figure*}
 \begin{center}$
  \centering
  \begin{array}{cc}
  \includegraphics[width=9.0cm]{figures/N6822_24vs3AT_Z} &
  \includegraphics[width=9.0cm]{figures/N6822_24vs3AT_Teff} \\
  \includegraphics[width=9.0cm]{figures/N6822_24vs3AT_logg} &
  \includegraphics[width=9.0cm]{figures/N6822_24vs3AT_Xi} \\
  \end{array}$
 \end{center}
 \caption{
            Comparison of the final model parameters using the two different telluric methods.
            Top left: metallicity ([Z]), mean difference
            $<\Delta[Z]>~= 0.04 \pm 0.07$.
            Top right: effective temperature (T$_{eff}$), mean difference
            $<\Delta T_{eff}>~= -14 \pm 42$.
            Bottom left: surface gravity (log $g$), mean difference
            $<\Delta$ log\,$g>~= -0.06 \pm 0.12$.
            Bottom right: Microturbulence ($\xi$), mean difference
            $<\Delta \xi>~= -0.1 \pm 0.1$.
            Black solid lines indicates direct correlation between the two methods.
            Green dashed lines indicates linear best fit to the data.
            In all cases, the distributions are statistically consistent with a one-to-one ratio (black lines).
          }
 \label{fig:3vs24AT}
\end{figure*}


\begin{table*}
\begin{center}
  % \centering
\caption{
Fit parameters for reductions using two different telluric methods.
\label{tb:stellar-params}
         }
\scriptsize
\begin{tabular}{lc cccc c cccc}
 \hline
 \hline
  Target  & IFU &  \multicolumn{4}{c}{24 Arm Telluric} & \multicolumn{4}{c}{3 Arm Telluric}\\
  \cline{3-6}  \cline{8-11}
 &  & T$_{fit}$ (K) & log g & $\xi$ (\kms) & [Z] & & $T_{fit}$ (K) & log g & $\xi$ (\kms) & [Z]\\
  \hline
RSG5  & 6 & 3790 $\pm$ 80\o & $-$0.0 $\pm$ 0.3 & 3.5 $\pm$ 0.4 & $-$0.55 $\pm$ 0.18 & & 3860 $\pm$ 90\o & $-$0.1 $\pm$ 0.5 &  3.5 $\pm$ 0.4 & $-$0.61 $\pm$ 0.21 \\
RSG8  & 11& 3850 $\pm$ 100  & \pp0.4 $\pm$ 0.5 & 3.5 $\pm$ 0.4 & $-$0.78 $\pm$ 0.22 & & 3810 $\pm$ 110  & \pp0.4 $\pm$ 0.5 &  3.3 $\pm$ 0.5 & $-$0.65 $\pm$ 0.24 \\
RSG12 & 12& 3880 $\pm$ 70\o & \pp0.0 $\pm$ 0.3 & 4.0 $\pm$ 0.4 & $-$0.32 $\pm$ 0.16 & & 3880 $\pm$ 70\o & \pp0.0 $\pm$ 0.3 &  4.0 $\pm$ 0.4 & $-$0.32 $\pm$ 0.16 \\
RSG24 & 2 & 3970 $\pm$ 60\o & \pp0.4 $\pm$ 0.5 & 3.9 $\pm$ 0.4 & $-$0.58 $\pm$ 0.19 & & 3990 $\pm$ 80\o & \pp0.1 $\pm$ 0.5 &  3.8 $\pm$ 0.5 & $-$0.56 $\pm$ 0.14 \\
RSG25 & 3 & 3910 $\pm$ 100  & \pp0.6 $\pm$ 0.5 & 3.0 $\pm$ 0.4 & $-$0.58 $\pm$ 0.24 & & 3910 $\pm$ 100  & \pp0.6 $\pm$ 0.5 &  3.0 $\pm$ 0.4 & $-$0.58 $\pm$ 0.24 \\
RSG29 & 4 & 3980 $\pm$ 60\o & \pp0.1 $\pm$ 0.4 & 3.7 $\pm$ 0.4 & $-$0.38 $\pm$ 0.16 & & 3990 $\pm$ 80\o & $-$0.1 $\pm$ 0.5 &  3.6 $\pm$ 0.4 & $-$0.44 $\pm$ 0.17 \\
RSG30 & 14& 3900 $\pm$ 80\o & $-$0.3 $\pm$ 0.5 & 3.7 $\pm$ 0.4 & $-$0.67 $\pm$ 0.16 & & 3850 $\pm$ 80\o & $-$0.3 $\pm$ 0.5 &  3.5 $\pm$ 0.4 & $-$0.59 $\pm$ 0.19 \\
RSG34 & 15& 3870 $\pm$ 80\o & $-$0.4 $\pm$ 0.5 & 4.2 $\pm$ 0.5 & $-$0.53 $\pm$ 0.19 & & 3850 $\pm$ 60\o & $-$0.3 $\pm$ 0.4 &  4.3 $\pm$ 0.5 & $-$0.49 $\pm$ 0.17 \\
RSG40 & 17& 3910 $\pm$ 110  & $-$0.5 $\pm$ 0.5 & 3.6 $\pm$ 0.5 & $-$0.20 $\pm$ 0.21 & & 3880 $\pm$ 110  & $-$0.5 $\pm$ 0.5 &  3.6 $\pm$ 0.5 & $-$0.15 $\pm$ 0.24 \\
RSG49 & 21& 3890 $\pm$ 120  & \pp0.1 $\pm$ 0.5 & 3.0 $\pm$ 0.4 & $-$0.43 $\pm$ 0.28 & & 3890 $\pm$ 120  & \pp0.1 $\pm$ 0.5 &  3.0 $\pm$ 0.4 & $-$0.43 $\pm$ 0.28 \\
RSG55 & 18& 3810 $\pm$ 130  & \pp0.4 $\pm$ 0.5 & 2.2 $\pm$ 0.4 & $-$0.68 $\pm$ 0.31 & & 3740 $\pm$ 130  & \pp0.4 $\pm$ 0.5 &  2.1 $\pm$ 0.5 & $-$0.54 $\pm$ 0.41 \\
% RSG9  & 5 & 4040 $\pm$ 130  & \pp0.8 $\pm$ 0.3 & 3.8 $\pm$ 0.6 & \pp0.14 $\pm$ 0.24 & 2500 & & 3990 $\pm$ 130  & \pp0.8 $\pm$ 0.3 &  3.8 $\pm$ 0.6 & \pp0.09 $\pm$ 0.24 & 2500 \\
% RSG16 & 7 & 4200 $\pm$ 50\o & \pp0.5 $\pm$ 0.4 & 2.3 $\pm$ 0.3 & $-$0.51 $\pm$ 0.16 & 4400 & & 4230 $\pm$ 100  & \pp0.5 $\pm$ 0.2 &  2.4 $\pm$ 0.3 & $-$0.54 $\pm$ 0.15 & 4400 \\
% RSG21 & 10& 3780 $\pm$ 110  & \pp0.4 $\pm$ 0.5 & 2.3 $\pm$ 0.4 & $-$0.55 $\pm$ 0.32 & 3800 & & 3790 $\pm$ 90\o & \pp0.5 $\pm$ 0.5 &  2.2 $\pm$ 0.4 & $-$0.46 $\pm$ 0.33 & 3800 \\
% RSG36 & 1 & 3840 $\pm$ 120  & \pp0.6 $\pm$ 0.5 & 3.0 $\pm$ 0.4 & $-$0.52 $\pm$ 0.26 & 3300 & & 3930 $\pm$ 100  & \pp0.6 $\pm$ 0.4 &  2.9 $\pm$ 0.4 & $-$0.39 $\pm$ 0.25 & 3300 \\
% RSG39 & 19& 3870 $\pm$ 130  & \pp0.5 $\pm$ 0.5 & 2.4 $\pm$ 0.5 & $-$0.73 $\pm$ 0.26 & 3000 & & 3800 $\pm$ 120  & \pp0.4 $\pm$ 0.5 &  2.1 $\pm$ 0.5 & $-$0.58 $\pm$ 0.33 & 3000 \\
% RSG45 & 24& 4220 $\pm$ 120  & \pp0.6 $\pm$ 0.5 & 2.9 $\pm$ 0.4 & $-$0.24 $\pm$ 0.20 & 3200 & & 4290 $\pm$ 120  & \pp0.6 $\pm$ 0.5 &  3.0 $\pm$ 0.5 & $-$0.27 $\pm$ 0.22 & 3200 \\
% RSG47 & 22& 3980 $\pm$ 90\o & \pp0.4 $\pm$ 0.4 & 3.2 $\pm$ 0.4 & $-$0.57 $\pm$ 0.23 & 2700 & & 3950 $\pm$ 100  & \pp0.4 $\pm$ 0.4 &  3.3 $\pm$ 0.4 & $-$0.64 $\pm$ 0.24 & 2700 \\

  \hline
  \end{tabular}
  \end{center}
\end{table*}

\subsection{Stellar Parameters and Metallicity} % (fold)
\label{sub:stellar_parameters_and_metallicity}

Table~\ref{tb:stellar-params} summarises the derived stellar parameters.
For the remainder of this paper, when discussing stellar parameters,
we exclusively use the parameters derived using the 24-arm telluric method (i.e. the left-hand results in Table~\ref{tb:stellar-params}.)

The average metallicity for our sample of 11 RSGs in NGC\,6822 is
$\bar{Z} = -0.52\pm 0.21$.
This result is in good agreement with the average metallicity derived in
NGC\,6822 from blue supergiant stars (BSGs)
\citep{1999A&A...352L..40M,2001ApJ...547..765V}.
Excluding the potential outlier (RSG5) from this analysis does not affect the average metallicity measurement.

A direct comparison between the metallicities of the objects mentioned above is legitimate as results derived from the analysis used here yield a global metallicity ([Z]) which
closely resembles the iron to hydrogen ratio ([Fe/H]) derived using BSGs.
While our [Z] measurements are also affected by Si, Mg and Ti,
we assume [Z]~=~[Fe/H] for the purposes of our discussion.
Likewise, metallicity measurements derived using H\2 regions yields oxygen abundances which can be compared
to [Z] by introducing the solar oxygen abundance
{12+log(O/H)}$_{\odot}$=8.69
\citep{2009ARA&A..47..481A} through the relation
[Z]=12+log(O/H)$-$8.69.

Physically, the metallicities derived using RSGs, BSGs and H\2 regions are comparable as these objects represent similar stellar populations within this galaxy.
The RSG and BSG stage are different evolutionary stages within the life cycle of a massive star.
H\2 regions, however, are the birth clouds which give rise to the most current stellar populations.
As we know the lifetime of RSGs/BSGs is $<50$Myr,
metallicity measurements from these stars are expected to represent the metallicities of their birth clouds.

To investigate spatial variations in chemical abundances in NGC\,6822, we show the metallicities of our RSGs as a function of radial distance from the centre of the galaxy 
Figure~\ref{fig:ZvsR}.

A least-squares fit to the data reveals a low-significance abundance gradient within the central 1\,kpc of NGC\,6822 of $-0.52\pm0.35$dex/kpc
The central metallicity (i.e. where R=0) of [Z]$=-0.3\pm0.15$ derived using the the least-squares fit remains consistent with the average metallicity assuming no gradient.

\begin{figure}
\includegraphics[width=9.0cm]{figures/N6822_ZvsR_BSG}
\caption{
Derived metallicities for 11 RSGs in NGC\,6822 shown against their distance from the galaxy centre.
The average metallicity for is
$\bar{Z} = -0.52\pm 0.21$.
A least-squares fit to the data reveals a low-significance abundance gradient
[Z]~=~$(-0.30\pm0.15)+(-0.52\pm0.35)R$ with a $\chi^{2}_{red}=1.14$.
Black points show metallicities derived in this study.
Blue points show the results from two A-type supergiant stars from
\protect\cite{2001ApJ...547..765V}.
Red points show the results from three BSG from
\protect\cite{1999A&A...352L..40M}.
Including these results into the fit we obtain a slightly shallower gradient
$-0.35\pm0.28$ dex\,kpc$^{-1}$ with an improved $\chi^{2}_{red}=0.72$.
R25 $= 460 \arcsec$ ($=1.12kpc$)
\citep{2012AJ....144....4M}.
Note,
\cite{1999A&A...352L..40M} do not quote individual metallicity values therefore the quoted average value is the value adopted for all.
        }
\label{fig:ZvsR}
\end{figure}

Figure~\ref{fig:6822HRD} shows the location of RSGs on the H-R diagram for RSGs.
Bolometric corrections are computed using the calibration in
\cite{Davies13a}.
This figure shows that the temperatures derived using the J-band analysis method are systematically cooler than the end of the evolutionary models for
$Z=0.002$
\citep{2013A&A...558A.103G}.
This is discussed in Section~\ref{sub:temperatures_of_rsgs}.


\begin{figure}
\includegraphics[width=9.0cm]{figures/N6822_HRD}
\caption{
H-R diagram (HRD) for 11 NGC\,6822 RSGs.
Evolutionary tracks including rotation
($v/v_{c} = 0.4$) for SMC-like metallicity ($Z=0.002$) are shown in grey with
along with their zero-age mass
\protect\citep{2013A&A...558A.103G}.
Bolometric corrections are computed using the calibration in
\protect\cite{Davies13a}.
We note that compared to the evolutionary tracks,
the observed temperatures of NGC\,6822 RSGs are systematically cooler.
This is discussed in Section~\ref{sub:temperatures_of_rsgs}.
}
\label{fig:6822HRD}
\end{figure}

% subsection stellar_parameters (end)
% section results (end)


\section{Discussion} % (fold)
\label{sec:discussion}

\subsection{Observing Strategy} % (fold)
\label{sub:observing_strategy}

Throughout these observations we used a O,\,S,\,O,\,O observing strategy.
However, in eight cases the sky subtraction process left weak residual features in the reduced spectra.
Reducing these cases with the \textquoteleft sky\_tweak\textquoteright
~option within the KMOS/esorex reduction pipeline was ineffective to improve the subtraction of these features.
Potential causes for these uncorrected features could include the changing intensity of the sky lines between science and sky exposures,
which could be solved by decreasing integration time.
We are currently developing an additional cross IFU sky subtraction routine which could alleviate this issue.


Based on the results of our comparison between the two different telluric methods with KMOS,
it is sufficient for our analysis to use the more efficient 3-arm method.
However, due to concerns around the sky subtraction and residuals from this process,
we caution against using this approach until a method with which we can standardise the resolution across each IFU is implemented.
We have shown that the stellar parameters derived with KMOS,
at a resolution of $\sim$3000 and a S/N $\ge$ 100,
are stable with respect to the choice of telluric spectrum for 11 of our observed RSGs.

% subsection observing_strategy (end)
\subsection{Metallicity Measurements} % (fold)
\label{sub:metallicity_measurements}

We find an average metallicity for NGC\,6822 of $\bar{Z}=-0.52\pm 0.21$
which agrees well with the results derived from BSGs
\citep{1999A&A...352L..40M,2001ApJ...547..765V,Przybilla02} and H\2 regions
\citep{2006ApJ...642..813L} in NGC\,6822.

We also find evidence for a low-significance metallicity gradient within the central 1\,kpc in NGC\,6822
($-0.52\pm0.35$ dex\,kpc$^{-1}$; see Figure~\ref{fig:ZvsR}).
The gradient derived is consistent with the trend reported in
\cite{2001ApJ...547..765V} who derived metallicities of BSGs and combined that with some of the best available H\2 region data.
The gradient is also consistent with the metallicity gradient derived from a sample of 49 local star-forming galaxies (Ho et al. in prep). 
\textbf{Include in reference list!}
Including the results for BSGs from
\cite{2001ApJ...547..765V} and
\cite{1999A&A...352L..40M} in our analysis,
results in a somewhat shallower gradient
($-0.35\pm 0.28$ dex\,kpc$^{-1}$)
with a slightly reduced 
$\chi^{2}_{red}=0.72$.

A discussion on why we are able to directly compare these results using different objects and yielding different tracers of metallicity (i.e. [Z], [Fe/H] and [O/H]) is given in
Section~\ref{sub:stellar_parameters_and_metallicity}.

\cite{2006ApJ...642..813L} use various H\2
regions and find no clear evidence for a metallicity gradient.
Using a consistent subset of the highest quality H\2
region data available these authors find a gradient of
$-0.16\pm0.05$dex\,kpc$^{-1}$.
Including these results into our analysis degrades the fit and changes the derived gradient significantly
($-0.18\pm0.05$ dex\,kpc$^{-1}$; $\chi^{2}_{red}=1.78$).
At this point it is not clear whether the indication of a gradient obtained from the RSGs and BSGs is just an artefact of the small sample size or indicates a difference with respect to the
H\2 region study.

\cite{2012A&A...540A.135S} derive metallicities using a population of asymptotic giant branch (AGB) stars within the central 4\,kpc of NGC\,6822.
The find an average metallicity of [Fe/H] $=-1.29\pm0.07$dex.
Likewise
\cite{2013ApJ...779..102K}
use spectra of red giant stars within the central 2\,kpc and find an average metallicity of
[Fe/H] $=-1.05\pm0.49$.
These authors find no compelling evidence for spatial variations in metallicity.
The stellar populations used for these studies are known to be significantly older than the RSGs,
therefore, owing to the chemical evolution in the time between the birth of these populations, we expect the measured metallicities to be significantly lower.
Additionally we expect a shallower metallicity gradient in these populations as the stellar motions act to smooth out abundance gradients over time.
Therefore it is unsurprising that these authors see no evidence for abundance gradients.

The low metallicity of the young stellar population and the ISM in NGC 6822 can be easily understood as a consequence of the fact that it is a relatively gas rich galaxy with a mass
M$_{HI}$ = 1.38$\times$10$^{8}$ M$_{\odot}$
\citep{2004AJ....128...16K} and a total stellar mass of
M$_{*}$ = 1$\times$10$^{8}$ M$_{\odot}$
\citep{2014ApJ...789..147W}.
The simple closed box chemical evolution model relates the metallicity mass fraction Z(t) at any time to the ratio of stellar to gas mass $M_{*}\over M_{g}$ through

\begin{equation}\label{closed-box}
Z(t) = {y \over 1-R } \ln \left[ 1 + {M_{*}(t)\over M_{g}(t)}  \right],
\end{equation}

where $y$ is the fraction of metals per stellar mass produced through stellar nucleosynthesis
(the so-called yield) and R is the fraction of stellar mass returned to the ISM through stellar mass-loss.
According to Kudritzki et al. (2014, in preparation) the ratio y/(1-R) can be empirically determined from the fact that the metallicity of the young stellar population in the solar neighborhood is solar with a mass fraction Z$_{\odot}$ =0.014
\citep{2012A&A...539A.143N}.
% (Nieva \& Przybilla, 2012).
With a solar neighborhood ratio of stellar to gas mass column densities of 4.48
\citep{2003ApJ...587..278W,2013ApJ...779..115B}
one then obtains y/(1-R) = 0.0082 = 0.59Z$_{\odot}$ with an uncertainty of 15 percent dominated by the the 0.05 dex uncertainty of the metallicity determination of the young population in the solar neighbourhood.
Accounting for the presence of helium and metals in the neutral interstellar gas we can turn the observed HI mass in NGC 6822 into gas mas via M$_{g}$ = 1.36 M$_{HI}$ and use the simple closed box model to predict a metallicity of [Z] = $-0.6\pm0.05$ in good agreement with our value obtained from RSG spectroscopy.

As discussed above the older stellar population of AGB stars has a metallicity roughly 0.7 dex lower than what we measure for the RSGs. In the framework of the simple closed box model this would correspond to a period in time where the ratio of stellar to gas mass was 0.07
(much lower than the present value of 0.53) and the stellar mass was only
$0.19\times10^{8}$M$_{\odot}$.
The present star formation rate of NGC 6822 is 0.027 M$_{\odot}$yr$^{-1}$
\citep{1996A&A...308..723I,2006ApJ...652.1170C,2010A&A...512A..68G}.
% (Gratier et al., 2010, Cannon et al., 2006, Israel et al., 1996).
At such a high level of star formation it would have taken three Gyr to produce the presently observed stellar mass and to arrive from the average AGB-star metallicity level at the metallicity of the young stellar population, of course, again relying on the simple closed box model.
With a lower star formation rate it would have taken correspondingly longer.

Given the irregularities present in the morphology of NGC\,6822 this galaxy may not be a good example of a closed box system, however it is remarkable that the closed box model reproduces the observed metallicity so closely.

% \cite{2013ApJ...779..102K} use two variations of the closed box chemical evolution model and find the model which represents the data best to be the model including a gas accretion term, the so called accertion model.
% However, these authors do note that dwarf irregular galaxies in general can be explaind by the more simple leaky box model. 

%%%%%%%%%%%%%%%%%%%%%%%%%%
%%% Not sure if I should comment on this ... 
%%% Question Chris/Rolf?
%%%%%%%%%%%%%%%%%%%%%%%%%%


% subsection abundance_measurements (end)

\subsection{Temperatures of RSGs} % (fold)
\label{sub:temperatures_of_rsgs}

Effective temperatures have been derived for 11 RSGs from our observed sample in NGC\,6822.
To date, this represents the fourth data set used to derive stellar parameters in this way and the first with KMOS.
The previous three data sets which have been analysed in this way are those of 11 RSGs in PerOB1,
a Galactic star cluster
\citep{2014ApJ...788...58G}, nine RSGs in the LMC and 10 RSGs in the SMC
\citep[both from][]{Davies14}.
These results span a range of $\sim$0.7dex in metallicity ranging from Z=Z$_{\odot}$ in PerOB1 to Z=0.3Z$_{\odot}$ in the SMC.

We compare the effective temperatures derived in this study to those of the previous results in different environments.
Their distribution is shown as a function of metallicity in Figure~\ref{fig:TvsZ}.
Additionally, Figure~\ref{fig:HRD} shows the H-R diagram for the same data set.
As mentioned above, stellar parameters have been derived in a consistent way for this data set. 
Bolometric corrections for the entire sample are computed using the calibration in
\cite{Davies13a}.


\begin{figure}
\includegraphics[width=9.0cm]{figures/N6822_TeffvsZ_all}
\caption{
Effective temperatures shown as a function of metallicity for four different data sets using the J-band analysis technique.
We show that there appears to be no significant evolution in the temperatures of RSGs over a range of 0.7 dex.
These data sets are compiled from the LMC, SMC
\protect\citep[blue and red points respectively;][]{Davies14}, PerOB1
\protect\citep[a Galactic RSG cluster; cyan points;][]{2014ApJ...788...58G} and NGC\,6822
(green points).
Mean values for each data set are enlarged data points in the same style.
The x-axis is reversed for comparison with Figure~\ref{fig:HRD}.\label{fig:TvsZ}
         }
\end{figure}

\begin{figure}
\includegraphics[width=9.0cm]{figures/N6822_HRD_all}
\caption{
H-R diagram (HRD) for red supergiants in the LMC, SMC and NGC\,6822 which have stellar parameters obtained using the J-band method.
This figure shows that there appears to be no significant temperature evolution of RSGs between the three studies.
NGC\,6822 targets from this study are shown with green circles.
LMC and SMC RSGs from
\protect\cite{Davies14}
are shown with blue triangles and red squares, respectively.
Solid grey lines show SMC-like metallicity evolutionary models including rotation
\protect\citep{2013A&A...558A.103G}.
Dashed grey lines show solar metallicity evolutionary models including rotation
\protect\citep{Ekstrom12}.\label{fig:HRD}
        }
\end{figure}


From these figures, we see no evidence for an variation in average temperatures of RSGs with respect to metallicity.
This is in contrast to current evolutionary models which display a change of $\sim$450K,
for a $M=15M_{\odot}$ model,
over a range of 0.7dex~\citep{Ekstrom12,2013A&A...558A.103G}.

For solar metallicity models, observations in PerOB1 are in good agreement with the models
\citep[see Figure 9 in][]{2014ApJ...788...58G}.
However, at SMC-like metallicity, the end-points of the evolutionary models are systematically warmer than the observations.
The temperature of the end-points of the evolutionary models of massive stars could depends on the choice of convective mixing length parameter
\citep{1992A&AS...96..269S}.
The observation that the models produce a lower temperature could imply that the choice of a solar-like mixing length parameter does not hold for higher mass stars at lower metallicity.

There is evidence however,
that the average spectral type of RSGs tends towards an earlier spectral type with decreasing metallicity over this range
\citep{Levesque12}.
We argue that these conclusions are not mutually exclusive.
Spectral types are derived for RSGs using the optical TiO band-heads at
$\sim$0.65$\mu$m,
whereas in this study temperatures are derived using near-IR atomic features
(as well as the line-free pseudo-continuum).
Recently,
\cite{Davies13a} showed that the strength of TiO bands are dependent upon metallicity and that at lower metallicity, the TiO bands are significantly weaker.
Therefore, although historically spectral type has been used as a proxy for temperature this assumption does not provide an accurate picture for RSGs.

% The trend in spectral type (or strength of the TiO absorption features) can be naturally explained by the decreasing abundance of the TiO molecule in lower metallicity environments.

% subsection temperatures_of_rsgs (end)

% section discussion (end)

\section{Conclusions} % (fold)
\label{sec:conclusions}

KMOS spectroscopic observations of red supergiant stars (RSGs) in NGC\,6822 are presented.
The data from these stars is telluric corrected in two different ways and the standard KMOS 3-arm telluric reduction is shown to work as effectively (in most cases) as the more time expensive 24-arm telluric reduction.

Stellar parameters are calculated for 11 RSGs using the J-band analysis method outlined in
\cite{Davies10}.
The average metallicity within NGC\,6822 is
$\bar{Z} = -0.52\pm 0.21$.
We find an indication of a metallicity gradient within the central 1\,kpc,
however with a low significance caused by small size of our RSG sample.
Stellar metallicity measurements from previous studies of young stars within this galaxy are compared and show that these data are consistent with our conclusions.
We compare the derived metallicity gradient to that of previous studies and find consistent results.
Using a closed box chemical evolution model,
measurements of the young and old stellar populations of NGC\,6822 can be explained through chemical evolution. 
We note that the closed box model is unlikely to be a good assumption for this galaxy given its morphology.
To conclusively determine the metallicity gradient among the young population within NGC\,6822 a larger systematic study of RSGs using this analysis is needed.

The effective temperatures of RSGs in this study are compared to those of all RSGs analysed in the same way.
Using a data set which spans 0.7 dex in metallicity (solar-like to SMC-like) within four galaxies, we find no evidence for a systematic variation in average effective temperature with respect to metallicity.
This is in contrast with evolutionary models for which a shift in metallicity of 0.7 dex produces a shift in the temperature of RSGs of up to 450K.
We argue that an observed shift in average spectral type of RSGs observed over the same metallicity range (0.7dex) does not imply a shift in average temperature.

These observations were taken as part of the KMOS Science Verification program.
With guaranteed time observations we have obtained data for RSGs in NGC\,300 and NGC\,55 at distances of $\sim$1.9\,Mpc,
as well as observations of super-star clusters in M\,83 and the Antennae galaxy at 4.5 and 20\,Mpc respectively.
Owing to the fact that RSGs dominate the light output from super-star clusters
\citep{2013MNRAS.430L..35G} these clusters can be analysed for abundance estimates in a similar manner
\citep{2014ApJ...787..142G},
which will provide metallicity measurements at distances a factor of 10 larger than using individual RSGs!
This project forms the basis of an ambitious general observation proposal to survey a large number of galaxies in the Local Volume,
motivated by the twin goals of investigating their abundance patterns,
while also calibrating the relationship between galaxy mass and metallicity in the Local Group.

% section conclusions (end)


%% If you wish to include an acknowledgments section in your paper,
%% separate it off from the body of the text using the \acknowledgments
%% command.

%% Included in this acknowledgments section are examples of the
%% AASTeX hypertext markup commands. Use \url without the optional [HREF]
%% argument when you want to print the url directly in the text. Otherwise,
%% use either \url or \anchor, with the HREF as the first argument and the
%% text to be printed in the second.

\acknowledgments

We thank Mike Irwin for providing the photometric catalogue from the WFCAM observations.
RPK and JZG acknowledge support by the National Science Foundation under grant AST-1108906
%% To help institutions obtain information on the effectiveness of their
%% telescopes, the AAS Journals has created a group of keywords for telescope
%% facilities. A common set of keywords will make these types of searches
%% significantly easier and more accurate. In addition, they will also be
%% useful in linking papers together which utilize the same telescopes
%% within the framework of the National Virtual Observatory.
%% See the AASTeX Web site at http://aastex.aas.org/
%% for information on obtaining the facility keywords.

%% After the acknowledgments section, use the following syntax and the
%% \facility{} macro to list the keywords of facilities used in the research
%% for the paper.  Each keyword will be checked against the master list during
%% copy editing.  Individual instruments or configurations can be provided
%% in parentheses, after the keyword, but they will not be verified.

{\it Facilities:}, \facility{VLT (KMOS)}.

%% Appendix material should be preceded with a single \appendix command.
%% There should be a \section command for each appendix. Mark appendix
%% subsections with the same markup you use in the main body of the paper.

%% Each Appendix (indicated with \section) will be lettered A, B, C, etc.
%% The equation counter will reset when it encounters the \appendix
%% command and will number appendix equations (A1), (A2), etc.

%% The reference list follows the main body and any appendices.
%% Use LaTeX's thebibliography environment to mark up your reference list.
%% Note \begin{thebibliography} is followed by an empty set of
%% curly braces.  If you forget this, LaTeX will generate the error
%% "Perhaps a missing \item?".
%%
%% thebibliography produces citations in the text using \bibitem-\cite
%% cross-referencing. Each reference is preceded by a
%% \bibitem command that defines in curly braces the KEY that corresponds
%% to the KEY in the \cite commands (see the first section above).
%% Make sure that you provide a unique KEY for every \bibitem or else the
%% paper will not LaTeX. The square brackets should contain
%% the citation text that LaTeX will insert in
%% place of the \cite commands.

%% We have used macros to produce journal name abbreviations.
%% AASTeX provides a number of these for the more frequently-cited journals.
%% See the Author Guide for a list of them.

%% Note that the style of the \bibitem labels (in []) is slightly
%% different from previous examples.  The natbib system solves a host
%% of citation expression problems, but it is necessary to clearly
%% delimit the year from the author name used in the citation.
%% See the natbib documentation for more details and options.

% Work around for refernces. Will need to change this for submission
\bibliographystyle{mn2e}                      % The reference style
\bibliography{journals}


% \begin{thebibliography}{}
% \bibitem[Auri\`ere(1982)]{aur82} Auri\`ere, M.  1982, \aap,
%     109, 301
% \bibitem[Canizares et al.(1978)]{can78} Canizares, C. R.,
%     Grindlay, J. E., Hiltner, W. A., Liller, W., \&
%     McClintock, J. E.  1978, \apj, 224, 39
% \bibitem[Djorgovski \& King(1984)]{djo84} Djorgovski, S.,
%     \& King, I. R.  1984, \apjl, 277, L49
% \bibitem[Hagiwara \& Zeppenfeld(1986)]{hag86} Hagiwara, K., \&
%     Zeppenfeld, D.  1986, Nucl.Phys., 274, 1
% \bibitem[Harris \& van den Bergh(1984)]{har84} Harris, W. E.,
%     \& van den Bergh, S.  1984, \aj, 89, 1816
% \bibitem[H\`enon(1961)]{hen61} H\'enon, M.  1961, Ann.d'Ap., 24, 369
% \bibitem[Heiles \& Troland(2003)]{heiles03} Heiles, C. \& Troland, T. H., 2003, \apjs, preprint doi:10.1086/381753
% \bibitem[Kim, Ostricker, \& Stone(2003)]{kim03} Kim, W.-T.,  Ostriker, E., \& Stone, J. M., 2003, \apj, 599, 1157
% \bibitem[King(1966)]{kin66}  King, I. R.  1966, \aj, 71, 276
% \bibitem[King(1975)]{kin75}  King, I. R.  1975, Dynamics of
%     Stellar Systems, A. Hayli, Dordrecht: Reidel, 1975, 99
% \bibitem[King et al.(1968)]{kin68}  King, I. R., Hedemann, E.,
%     Hodge, S. M., \& White, R. E.  1968, \aj, 73, 456
% \bibitem[Kron et al.(1984)]{kro84} Kron, G. E., Hewitt, A. V.,
%     \& Wasserman, L. H.  1984, \pasp, 96, 198
% \bibitem[Lynden-Bell \& Wood(1968)]{lyn68} Lynden-Bell, D.,
%     \& Wood, R.  1968, \mnras, 138, 495
% \bibitem[Newell \& O'Neil(1978)]{new78} Newell, E. B.,
%     \& O'Neil, E. J.  1978, \apjs, 37, 27
% \bibitem[Ortolani et al.(1985)]{ort85} Ortolani, S., Rosino, L.,
%     \& Sandage, A.  1985, \aj, 90, 473
% \bibitem[Peterson(1976)]{pet76} Peterson, C. J.  1976, \aj, 81, 617
% \bibitem[Rudnick et al.(2003)]{rudnick03} Rudnick, G. et al., 2003, \apj, 599, 847
% \bibitem[Spitzer(1985)]{spi85} Spitzer, L.  1985, Dynamics of
%     Star Clusters, J. Goodman \& P. Hut, Dordrecht: Reidel, 109
% \bibitem[Treu et al.(2003)]{treu03} Treu, T. et al., 2003, \apj, 591, 53
% \end{thebibliography}

% \clearpage


%% The following command ends your manuscript. LaTeX will ignore any text
%% that appears after it.

\end{document}

%%
%% End of file
